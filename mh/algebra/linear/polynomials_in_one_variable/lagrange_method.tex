\subsection{Интерполяционный многочлен Лагранжа}
\index{интерполяционный многочлен Лагранжа}\textbf{Интерполяционный многочлен Лагранжа} — многочлен минимальной степени, принимающий данные значения в данном наборе точек.

Пусть даны $n+1$ пар чисел $(x_0, y_0)$, $(x_1, y_1)$, $\ldots$, $(x_n, y_n)$, тогда многочлен Лагранжа имеет вид 
$$
L(x) = \sum\limits_{i=0}^n(y_i \prod\limits_{j=0, j \neq i}^n \frac{x - x_j}{x_i - x_j})
$$

График многочлена Лагранжа содержит все $n+1$ точки, то есть $\forall 0 \leq i \leq n : L(x_i) = y_i$ 
\begin{proof}
	Возьмём из данного набора любую точку $(x_k, y_k)$. В формуле $L(x) = \sum\limits_{i=0}^n(y_i \prod\limits_{j=0, j \neq i}^n \frac{x - x_j}{x_i - x_j})$ у все слагаемых, кроме $k$-ого будет множитель $x - x_k$, поэтому при $x = x_k$ все эти слагаемые обнулятся и многочлен примет вид $L(x_k) = y_k \prod\limits_{j=0, j \neq k}^n \frac{x_k - x_j}{x_k - x_j} = y_k$.
\end{proof}

Существует единственный многочлен степени не превосходящей $n$, который принимает заданные значения в $n+1$ точке
\begin{proof}
	Путь существует два многочлена $P(x)$ и $Q(x)$ степени не больше $n$, которые принимают заданные значения в $n+1$ точке $(x_0, y_0)$, $(x_1, y_1)$, $\ldots$, $(x_n, y_n)$.
	
	Рассмотрим многочлен $T(x) = P(x) - Q(x)$. Его степень тоже не больше $n$. Подставим каждую точку из набора в этот многочлен, получим $T(x_i) = P(x_i) - Q(x_i) = y_i - y_i = 0$, то есть у $T(x)$ есть $n + 1$ корень, что противоречит тому, что ненулевой многочлен степени, не превосходящей $n$ имеет не более $n$ корней.
\end{proof}
