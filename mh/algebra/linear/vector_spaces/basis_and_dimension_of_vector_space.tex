% beta
\subsection{Базис и размерность векторного пространства}
Векторы $\overline{a_1}, \ldots, \overline{a_n}$ называются \textbf{линейно зависимыми}, если
\begin{equation*}
\exists \alpha_1, \ldots, \alpha_n \colon
\sum_{i=1}^n \alpha_i \overline{a_i} = \overline 0, \ 
\sum_{i=1}^n \alpha_i^2 \neq 0
\end{equation*}
иначе~--- \textbf{линейно независимыми}.

Множество линейно независимых векторов $\overline{e_1}, \ldots, \overline{e_n}$ векторного пространства~$V$ называется \textbf{базисом} этого пространства, если
\begin{equation*}
\forall \overline x \in V \ 
\exists \alpha_1, \ldots, \alpha_n \colon
\overline x = \sum_{i=1}^n \alpha_i \overline{e_i}
\end{equation*}

Приведённое равенство называется \textbf{разложением} вектора~$\overline x$ по~базису~$\overline{e_1}, \ldots, \overline{e_n}$.

\begin{theorem}[о базисе]
Любой вектор~$\overline x$ может быть разложен по~базису~$\overline{e_1}, \ldots, \overline{e_n}$ единственным образом.
\end{theorem}
\begin{proof}

\end{proof}

\textbf{Размерностью} векторного пространства называется максимальное количество линейно независимых векторов.

\begin{theorem}
В~векторном пространстве размерности $n$ любые $n$ линейно независимых векторов образуют его базис.
\end{theorem}
\begin{proof}

\end{proof}

\begin{theorem}
Если векторное пространство имеет базис из~$n$~векторов, то его размерность равна $n$.
\end{theorem}
\begin{proof}

\end{proof}