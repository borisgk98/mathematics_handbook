\section{Векторные пространства}
$n$\nobreakdash-мерным векторным пространством над~полем вещественных чисел называется множество
\begin{equation*}
V_n = \mathbb R^n = \{ (x_1; \ldots; x_n) \mid x_1, \ldots, x_n \in \mathbb R \}
\end{equation*}
элементы которого называются \textbf{векторами}. Над~ними определены операции сложения и~умножения на число, удовлетворяющие аксиомам:
\begin{enumerate}
	\item Коммутативность сложения:
	\begin{equation*}
	\forall \overline a, \overline b \in V_n \ 
	\overline a + \overline b = \overline b + \overline a
	\end{equation*}
	
	\item Ассоциативность сложения:
	\begin{equation*}
	\forall \overline a, \overline b, \overline c \in V_n \ 
	\overline a + (\overline b + \overline c) = (\overline a + \overline b) + \overline c
	\end{equation*}
	
	\item Существование \textbf{нулевого} вектора, или~\textbf{нуля}, обозначаемого $\overline 0$:
	\begin{equation*}
	\exists \overline 0 \in V_n \colon \forall \overline a \in V \ 
	\overline a + \overline 0 = \overline 0 + \overline a = \overline a
	\end{equation*}
	
	\item Существование \textbf{противоположного} вектора:
	\begin{equation*}
	\forall \overline a \in V_n \ 
	\exists (-\overline a) \in V_n \colon
	\overline a + (-\overline a) = \overline 0
	\end{equation*}
	
	\item Ассоциативность умножения на~число:
	\begin{equation*}
	\forall \alpha, \beta \in \mathbb R, \ 
	\forall \overline a \in V_n \ 
	\alpha (\beta \overline a) = (\alpha \beta) \overline a
	\end{equation*}
	
	\item Дистрибутивность умножения на~число относительно сложения векторов:
	\begin{equation*}
	\forall \alpha \in \mathbb R, \ 
	\forall \overline a, \overline b \in V_n \ 
	\alpha (\overline a + \overline b) = \alpha \overline a + \alpha \overline b
	\end{equation*}
	
	\item Дистрибутивность умножения на~число относительно сложения чисел:
	\begin{equation*}
	\forall \alpha, \beta \in \mathbb R, \ 
	\forall \overline a \in V_n \ 
	(\alpha + \beta) \overline a = \alpha \overline a + \beta \overline a
	\end{equation*}
	
	\item Существование \textbf{единицы}:
	\begin{equation*}
	\forall \overline a \in V_n \ 
	1 \cdot \overline a = \overline a
	\end{equation*}
\end{enumerate}


% beta
\subsection{Базис и размерность векторного пространства}
Векторы $\overline{a_1}, \ldots, \overline{a_n}$ называются \textbf{линейно зависимыми}, если
\begin{equation*}
\exists \alpha_1, \ldots, \alpha_n \colon
\sum_{i=1}^n \alpha_i \overline{a_i} = \overline 0, \ 
\sum_{i=1}^n \alpha_i^2 \neq 0
\end{equation*}
иначе~--- \textbf{линейно независимыми}.

Множество линейно независимых векторов $\overline{e_1}, \ldots, \overline{e_n}$ векторного пространства~$V$ называется \textbf{базисом} этого пространства, если
\begin{equation*}
\forall \overline x \in V \ 
\exists \alpha_1, \ldots, \alpha_n \colon
\overline x = \sum_{i=1}^n \alpha_i \overline{e_i}
\end{equation*}

Приведённое равенство называется \textbf{разложением} вектора~$\overline x$ по~базису~$\overline{e_1}, \ldots, \overline{e_n}$.

\begin{theorem}[о базисе]
Любой вектор~$\overline x$ может быть разложен по~базису~$\overline{e_1}, \ldots, \overline{e_n}$ единственным образом.
\end{theorem}
\begin{proof}

\end{proof}

\textbf{Размерностью} векторного пространства называется максимальное количество линейно независимых векторов.

\begin{theorem}
В~векторном пространстве размерности $n$ любые $n$ линейно независимых векторов образуют его базис.
\end{theorem}
\begin{proof}

\end{proof}

\begin{theorem}
Если векторное пространство имеет базис из~$n$~векторов, то его размерность равна $n$.
\end{theorem}
\begin{proof}

\end{proof}