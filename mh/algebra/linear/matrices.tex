\section{Матрицы}
\textbf{Матрицей} называется прямоугольная таблица из~чисел, содержащая $m$~строк и~$n$~столбцов, и~обозначается
\begin{equation*}
A =
\begin{pmatrix}
a_{11} & a_{12} & \cdots & a_{1n} \\
a_{21} & a_{22} & \cdots & a_{2n} \\
\vdots & \vdots & \ddots & \vdots \\
a_{m1} & a_{m2} & \cdots & a_{mn}
\end{pmatrix} =
\begin{Vmatrix}
a_{11} & a_{12} & \cdots & a_{1n} \\
a_{21} & a_{22} & \cdots & a_{2n} \\
\vdots & \vdots & \ddots & \vdots \\
a_{m1} & a_{m2} & \cdots & a_{mn}
\end{Vmatrix}
\end{equation*}

Числа $m$ и~$n$ называются \textbf{порядками} матрицы.

Если $m = n$, то матрица называется \textbf{квадратной}, а число~$m = n$~--- её \textbf{порядком}.
\textbf{Главной} называется диагональ квадратной матрицы, состоящая из~элементов $a_{11}, a_{22}, \ldots, a_{nn}$, а \textbf{побочной}~--- состоящая из~элементов $a_{n1}, a_{n-1\, 2}, \ldots, a_{1n}$.

$i$\nobreakdash-я строка матрицы обозначается~$A_i$, $j$\nobreakdash-й столбец~---~$A^j$.

Две матрицы называются \textbf{равными}, если их порядки и~соответствующие элементы совпадают, иначе~--- \textbf{неравными}.


\subsection{Операции над матрицами}
Матрица, все элементы которой равны 0, называется \textbf{нулевой} и~обозначается $O$.

Квадратная матрица, в~которой элементы главной диагонали равны 1, а остальные~--- 0, называется \textbf{единичной} и~обозначается $E$.

Над~матрицами определены следующие операции:
\begin{itemize}
	\item\textbf{Сложение.}
	Определено только над~матрицами одинакового размера.
	\begin{equation*}
	\begin{Vmatrix}
	a_{11} & a_{12} & \cdots & a_{1n} \\ 
	a_{21} & a_{22} & \cdots & a_{2n} \\ 
	\vdots & \vdots & \ddots & \vdots \\ 
	a_{m1} & a_{m2} & \cdots & a_{mn}
	\end{Vmatrix} +
	\begin{Vmatrix}
	b_{11} & b_{12} & \cdots & b_{1n} \\ 
	b_{21} & b_{22} & \cdots & b_{2n} \\ 
	\vdots & \vdots & \ddots & \vdots \\ 
	b_{m1} & b_{m2} & \cdots & b_{mn}
	\end{Vmatrix} =
	\begin{Vmatrix}
	a_{11} + b_{11} & a_{12} + b_{12} & \cdots & a_{1n} + b_{1n} \\ 
	a_{21} + b_{21} & a_{22} + b_{22} & \cdots & a_{2n} + b_{2n} \\ 
	\vdots & \vdots & \ddots & \vdots \\ 
	a_{m1} + b_{m1} & a_{m2} + b_{m2} & \cdots & a_{mn} + b_{mn}
	\end{Vmatrix}
	\end{equation*}
	
	Пусть $A, B, C$~--- матрицы. Свойства сложения:
	\begin{itemize}
		\item коммутативность:
		$A + B = B + A$
		\item ассоциативность:
		$(A + B) + C = A + (B + C)$
	\end{itemize}
	
	\item\textbf{Умножение на~число.}
	\begin{equation*}
	\lambda \cdot
	\begin{Vmatrix}
	a_{11} & a_{12} & \cdots & a_{1n} \\ 
	a_{21} & a_{22} & \cdots & a_{2n} \\ 
	\vdots & \vdots & \ddots & \vdots \\ 
	a_{m1} & a_{m2} & \cdots & a_{mn}
	\end{Vmatrix} =
	\begin{Vmatrix}
	\lambda a_{11} & \lambda a_{12} & \cdots & \lambda a_{1n} \\ 
	\lambda a_{21} & \lambda a_{22} & \cdots & \lambda a_{2n} \\ 
	\vdots & \vdots & \ddots & \vdots \\ 
	\lambda a_{m1} & \lambda a_{m2} & \cdots & \lambda a_{mn}
	\end{Vmatrix}
	\end{equation*}
	
	Пусть $\alpha, \beta$~--- числа, $A, B$~--- матрицы. Свойства умножения на~число:
	\begin{itemize}
		\item ассоциативность:
		$(\alpha \cdot \beta) \cdot A = \alpha \cdot (\beta \cdot A)$
		\item дистрибутивность относительно~сложения чисел:
		$(\alpha + \beta) \cdot A = \alpha \cdot A + \beta \cdot A$
		\item дистрибутивность относительно~сложения матриц:
		$\alpha \cdot (A + B) = \alpha \cdot A + \alpha \cdot B$
	\end{itemize}
	
	\item\textbf{Умножение}. $A \cdot B$ определено, только если количество столбцов в~матрице $A$ совпадает с~количеством строк в~матрице $B$.
	\begin{equation*}
	\begin{Vmatrix}
	a_{11} & a_{12} & \cdots & a_{1k} \\ 
	a_{21} & a_{22} & \cdots & a_{2k} \\ 
	\vdots & \vdots & \ddots & \vdots \\ 
	a_{m1} & a_{m2} & \cdots & a_{mk}
	\end{Vmatrix} \cdot
	\begin{Vmatrix}
	b_{11} & b_{12} & \cdots & b_{1n} \\ 
	b_{21} & b_{22} & \cdots & b_{2n} \\ 
	\vdots & \vdots & \ddots & \vdots \\ 
	b_{k1} & b_{k2} & \cdots & b_{kn}
	\end{Vmatrix} =
	\begin{Vmatrix}
	\sum a_{1i}b_{i1} & \sum a_{1i}b_{i2} & \cdots & \sum a_{1i}b_{in} \\
	\sum a_{2i}b_{i1} & \sum a_{2i}b_{i2} & \cdots & \sum a_{2i}b_{in} \\
	\vdots & \vdots & \ddots & \vdots \\
	\sum a_{mi}b_{i1} & \sum a_{mi}b_{i2} & \cdots & \sum a_{mi}b_{in}
	\end{Vmatrix}
	\end{equation*}
	где суммирование производится по~$i$ от~$1$ до~$k$.
	
	Пусть $\lambda$~--- число, $A, B, C$~--- матрицы. Свойства умножения:
	\begin{itemize}
		\item ассоциативность:
		$(A \cdot B) \cdot C = A \cdot (B \cdot C)$
		\item дистрибутивность:
		$(A + B) \cdot C \opbr= A \cdot C + B \cdot C$,
		$A \cdot (B + C) \opbr= A \cdot B + A \cdot C$
		\item ассоциативность и~коммутативность относительно~умножения на~число:
		$(\lambda \cdot A) \cdot B \opbr= \lambda \cdot (A \cdot B) \opbr= A \cdot (\lambda \cdot B)$
	\end{itemize}
\end{itemize}
\subsection{Определитель матрицы}
\index{Определитель матрицы}\textbf{Определителем} порядка $n$, соответствующим квадратной матрице $A$ порядка $n$, называется число, равное
\begin{equation}
\label{eq:determinant}
\Delta = \det A =
\begin{vmatrix}
a_{11} & a_{12} & \cdots & a_{1n} \\
a_{21} & a_{22} & \cdots & a_{2n} \\
\vdots & \vdots & \ddots & \vdots \\
a_{n1} & a_{n2} & \cdots & a_{nn}
\end{vmatrix} =
\sum_{\sigma = (i_1; \ldots; i_n) \in S_n} (-1)^{|\sigma|} a_{1\, i_1} a_{2\, i_2} \cdot \ldots \cdot a_{n\, i_n}, \ 
|\sigma| =
\begin{cases}
0, \sigma \text{ чётная} \\
1, \sigma \text{ нечётная}
\end{cases}
\end{equation}
где $S_n$~--- множество всех перестановок $n$\nobreakdash-\hspace{0pt}элементного множества.

Матрица называется \textbf{вырожденной}, если её определитель равен 0, иначе~--- \textbf{невырожденной}.

Свойства определителя:
\begin{itemize}
	\item Если элементы какой-либо строки или~столбца определителя имеют общий множитель $\lambda$, то его можно вынести за~знак определителя.
	\begin{proof}
	\begin{equation*}
	\Delta = \sum (-1)^{|\sigma|} a_{1\, i_1} a_{2\, i_2} \cdot \ldots \cdot a_{n\, i_n}
	\end{equation*}
	Каждое слагаемое имеет множитель из~каждой строки, а также из~каждого столбца, т.\,к. $\sigma$ является перестановкой и~содержит все номера столбцов от~$1$ до~$n$ включительно.
	Тогда все слагаемые имеют общий множитель $\lambda$, поэтому его можно вынести за~скобки.
	\end{proof}
	
	\item Если какая-либо строка или~столбец определителя состоит из~нулей, то он равен 0.
	
	\item \begin{equation*}
	\begin{vmatrix}
	a_{11} & a_{12} & \cdots & a_{1n} \\
	\vdots & \vdots & \ddots & \vdots \\
	a_{i1} + b_{i1} & a_{i2} + b_{i2} & \cdots & a_{in} + b_{in} \\
	\vdots & \vdots & \ddots & \vdots \\
	a_{n1} & a_{n2} & \cdots & a_{nn}
	\end{vmatrix} =
	\begin{vmatrix}
	a_{11} & a_{12} & \cdots & a_{1n} \\
	\vdots & \vdots & \ddots & \vdots \\
	a_{i1} & a_{i2} & \cdots & a_{in} \\
	\vdots & \vdots & \ddots & \vdots \\
	a_{n1} & a_{n2} & \cdots & a_{nn}
	\end{vmatrix} +
	\begin{vmatrix}
	a_{11} & a_{12} & \cdots & a_{1n} \\
	\vdots & \vdots & \ddots & \vdots \\
	b_{i1} & b_{i2} & \cdots & b_{in} \\
	\vdots & \vdots & \ddots & \vdots \\
	a_{n1} & a_{n2} & \cdots & a_{nn}
	\end{vmatrix}
	\end{equation*}
	Свойство для~столбцов аналогично.
	\begin{proof}
	\begin{equation*}
	\Delta = \begin{vmatrix}
	a_{11} & a_{12} & \cdots & a_{1n} \\
	\vdots & \vdots & \ddots & \vdots \\
	a_{i1} + b_{i1} & a_{i2} + b_{i2} & \cdots & a_{in} + b_{in} \\
	\vdots & \vdots & \ddots & \vdots \\
	a_{n1} & a_{n2} & \cdots & a_{nn}
	\end{vmatrix} =
	\sum (-1)^{|\sigma|} a_{1\, i_1} \cdot \ldots \cdot a_{n\, i_n} =
	\end{equation*}
	\begin{equation*}
	\left| \text{ Каждое слагаемое содержит ровно 1 элемент из~$i$\nobreakdash-й строки и~поэтому имеет вид } \right|
	\end{equation*}
	\begin{equation*}
	= \sum (-1)^{|\sigma|} a_{1\, i_1} \cdot \ldots \cdot a_{k-1\, i_{k-1}} (a_{k\, i_k} + b_{k\, i_k}) a_{k+1\, i_{k+1}} \cdot \ldots \cdot a_{n\, i_n} =
	\end{equation*}
	\begin{equation*}
	= \sum (-1)^{|\sigma|} a_{1\, i_1} \cdot \ldots \cdot a_{k\, i_k} \cdot \ldots \cdot a_{n\, i_n} +
	\sum (-1)^{|\sigma|} a_{1\, i_1} \cdot \ldots \cdot b_{k\, i_k} \cdot \ldots \cdot a_{n\, i_n} =
	\end{equation*}
	\begin{equation*}
	= \begin{vmatrix}
	a_{11} & a_{12} & \cdots & a_{1n} \\
	\vdots & \vdots & \ddots & \vdots \\
	a_{i1} & a_{i2} & \cdots & a_{in} \\
	\vdots & \vdots & \ddots & \vdots \\
	a_{n1} & a_{n2} & \cdots & a_{nn}
	\end{vmatrix} +
	\begin{vmatrix}
	a_{11} & a_{12} & \cdots & a_{1n} \\
	\vdots & \vdots & \ddots & \vdots \\
	b_{i1} & b_{i2} & \cdots & b_{in} \\
	\vdots & \vdots & \ddots & \vdots \\
	a_{n1} & a_{n2} & \cdots & a_{nn}
	\end{vmatrix}
	\end{equation*}
	Свойство для~столбцов доказывается аналогично.
	\end{proof}
	
	\item Если в~определителе поменять две строки или~два столбца местами, то он изменит знак.
	\begin{proof}
	При перестановке строк или~столбцов местами все перестановки в~формуле~(\ref{eq:determinant}) меняют чётность, значит, каждое слагаемое меняет знак, тогда и~определитель меняет знак.
	\end{proof}
	
	\item Если в~определителе две строки или~два столбца совпадают, то он равен 0.
	\begin{proof}
	Если поменять местами совпадающие строки или~столбцы, то он, с~одной стороны, не~изменится, а с~другой, поменяет знак. Значит, определитель равен 0.
	\end{proof}
	
	\item \begin{equation*}
	\begin{vmatrix}
	a_{11} & a_{12} & \cdots & a_{1n} \\
	\vdots & \vdots & \ddots & \vdots \\
	a_{i1} & a_{i2} & \cdots & a_{in} \\
	\vdots & \vdots & \ddots & \vdots \\
	a_{j1} & a_{j2} & \cdots & a_{jn} \\
	\vdots & \vdots & \ddots & \vdots \\
	a_{n1} & a_{n2} & \cdots & a_{nn}
	\end{vmatrix} =
	\begin{vmatrix}
	a_{11} & a_{12} & \cdots & a_{1n} \\
	\vdots & \vdots & \ddots & \vdots \\
	a_{i1} & a_{i2} & \cdots & a_{in} \\
	\vdots & \vdots & \ddots & \vdots \\
	\lambda a_{i1} + a_{j1} & \lambda a_{i2} + a_{j2} & \cdots & \lambda a_{in} + a_{jn} \\
	\vdots & \vdots & \ddots & \vdots \\
	a_{n1} & a_{n2} & \cdots & a_{nn}
	\end{vmatrix}
	\end{equation*}
	Свойство для~столбцов аналогично.
	\begin{proof}
	\begin{equation*}
	\begin{vmatrix}
	a_{11} & a_{12} & \cdots & a_{1n} \\
	\vdots & \vdots & \ddots & \vdots \\
	a_{i1} & a_{i2} & \cdots & a_{in} \\
	\vdots & \vdots & \ddots & \vdots \\
	a_{j1} & a_{j2} & \cdots & a_{jn} \\
	\vdots & \vdots & \ddots & \vdots \\
	a_{n1} & a_{n2} & \cdots & a_{nn}
	\end{vmatrix} =
	\begin{vmatrix}
	a_{11} & a_{12} & \cdots & a_{1n} \\
	\vdots & \vdots & \ddots & \vdots \\
	a_{i1} & a_{i2} & \cdots & a_{in} \\
	\vdots & \vdots & \ddots & \vdots \\
	\lambda a_{i1} & \lambda a_{i2} & \cdots & \lambda a_{in} \\
	\vdots & \vdots & \ddots & \vdots \\
	a_{n1} & a_{n2} & \cdots & a_{nn}
	\end{vmatrix} +
	\begin{vmatrix}
	a_{11} & a_{12} & \cdots & a_{1n} \\
	\vdots & \vdots & \ddots & \vdots \\
	a_{i1} & a_{i2} & \cdots & a_{in} \\
	\vdots & \vdots & \ddots & \vdots \\
	a_{j1} & a_{j2} & \cdots & a_{jn} \\
	\vdots & \vdots & \ddots & \vdots \\
	a_{n1} & a_{n2} & \cdots & a_{nn}
	\end{vmatrix} =
	\end{equation*}
	\begin{equation*}
	= \begin{vmatrix}
	a_{11} & a_{12} & \cdots & a_{1n} \\
	\vdots & \vdots & \ddots & \vdots \\
	a_{i1} & a_{i2} & \cdots & a_{in} \\
	\vdots & \vdots & \ddots & \vdots \\
	\lambda a_{i1} + a_{j1} & \lambda a_{i2} + a_{j2} & \cdots & \lambda a_{in} + a_{jn} \\
	\vdots & \vdots & \ddots & \vdots \\
	a_{n1} & a_{n2} & \cdots & a_{nn}
	\end{vmatrix}
	\end{equation*}
	Свойство для~столбцов доказывается аналогично.
	\end{proof}
\end{itemize}

Пусть дана матрица
\begin{equation*}
\begin{Vmatrix}
a_{11} & a_{12} & \cdots & a_{1n} \\
a_{21} & a_{22} & \cdots & a_{2n} \\
\vdots & \vdots & \ddots & \vdots \\
a_{n1} & a_{n2} & \cdots & a_{nn}
\end{Vmatrix}
\end{equation*}

\index{Алгебраическое дополнение}\textbf{Алгебраическим дополнением} элемента~$a_{ij}$ называется число, равное
\begin{equation*}
A_{ij} = (-1)^{i+j} \cdot
\begin{vmatrix}
a_{11} & \cdots & a_{1\, j-1} & a_{1\, j+1} & \cdots & a_{1n} \\
\vdots & \ddots & \vdots & \vdots & \ddots & \vdots \\
a_{i-1\, 1} & \cdots & a_{i-1\, j-1} & a_{i-1\, j+1} & \cdots & a_{i-1\, n} \\
a_{i+1\, 1} & \cdots & a_{i+1\, j-1} & a_{i+1\, j+1} & \cdots & a_{i+1\, n} \\
\vdots & \ddots & \vdots & \vdots & \ddots & \vdots \\
a_{n1} & \cdots & a_{n\, j-1} & a_{n\, j+1} & \cdots & a_{nn}
\end{vmatrix}
\end{equation*}

\begin{lemma}
\begin{equation*}
\begin{vmatrix}
a & 0 & \cdots & 0 \\
a_{21} & a_{22} & \cdots & a_{2n} \\
\vdots & \vdots & \ddots & \vdots \\
a_{n1} & a_{n2} & \cdots & a_{nn}
\end{vmatrix} = a \cdot
\begin{vmatrix}
a_{22} & \cdots & a_{2n} \\
\vdots & \ddots & \vdots \\
a_{n2} & \cdots & a_{nn}
\end{vmatrix}
\end{equation*}
\end{lemma}
\begin{proof}
\begin{equation*}
\begin{vmatrix}
a & 0 & \cdots & 0 \\
a_{21} & a_{22} & \cdots & a_{2n} \\
\vdots & \vdots & \ddots & \vdots \\
a_{n1} & a_{n2} & \cdots & a_{nn}
\end{vmatrix} =
\end{equation*}
\begin{equation*}
= \sum a \cdot a_{2\, i_2} \cdot \ldots \cdot a_{n\, i_n} +
\sum 0 \cdot a_{2\, i_2} \cdot \ldots \cdot a_{n\, i_n} + \ldots +
\sum 0 \cdot a_{2\, i_2} \cdot \ldots \cdot a_{n\, i_n} =
\end{equation*}
\begin{equation*}
= a \sum a_{2\, i_2} \cdot \ldots \cdot a_{n\, i_n} = a \cdot
\begin{vmatrix}
a_{22} & \cdots & a_{2n} \\
\vdots & \ddots & \vdots \\
a_{n2} & \cdots & a_{nn}
\end{vmatrix}
\end{equation*}
\end{proof}

\begin{theorem}
Любой определитель можно \textbf{разложить} по~элементам произвольной строки или~столбца:
\begin{equation*}
\begin{vmatrix}
a_{11} & a_{12} & \cdots & a_{1n} \\
a_{21} & a_{22} & \cdots & a_{2n} \\
\vdots & \vdots & \ddots & \vdots \\
a_{n1} & a_{n2} & \cdots & a_{nn}
\end{vmatrix}
= \sum_{j=1}^n a_{ij} A_{ij}
= \sum_{i=1}^n a_{ij} A_{ij}
\end{equation*}
где $A_{ij}$~--- алгебраическое дополнение элемента~$a_{ij}$.
\end{theorem}
\begin{proof}
\begin{equation*}
\begin{vmatrix}
a_{11} & a_{12} & \cdots & a_{1n} \\
\vdots & \vdots & \ddots & \vdots \\
a_{i1} & a_{i2} & \cdots & a_{in} \\
\vdots & \vdots & \ddots & \vdots \\
a_{n1} & a_{n2} & \cdots & a_{nn}
\end{vmatrix} = (-1)^{i-1} \cdot
\begin{vmatrix}
a_{i1} & a_{i2} & \cdots & a_{in} \\
a_{11} & a_{12} & \cdots & a_{1n} \\
\vdots & \vdots & \ddots & \vdots \\
a_{i-1\, 1} & a_{i-1\, 2} & \cdots & a_{i-1\, n} \\
a_{i+1\, 1} & a_{i+1\, 2} & \cdots & a_{i+1\, n} \\
\vdots & \vdots & \ddots & \vdots \\
a_{n1} & a_{n2} & \cdots & a_{nn}
\end{vmatrix} = (-1)^{i+1} \cdot
\end{equation*}
\begin{equation*}
\cdot \left(
\begin{vmatrix}
a_{i1} & 0 & \cdots & 0 \\
a_{11} & a_{12} & \cdots & a_{1n} \\
\vdots & \vdots & \ddots & \vdots \\
a_{i-1\, 1} & a_{i-1\, 2} & \cdots & a_{i-1\, n} \\
a_{i+1\, 1} & a_{i+1\, 2} & \cdots & a_{i+1\, n} \\
\vdots & \vdots & \ddots & \vdots \\
a_{n1} & a_{n2} & \cdots & a_{nn}
\end{vmatrix} +
\begin{vmatrix}
0 & a_{i2} & \cdots & 0 \\
a_{11} & a_{12} & \cdots & a_{1n} \\
\vdots & \vdots & \ddots & \vdots \\
a_{i-1\, 1} & a_{i-1\, 2} & \cdots & a_{i-1\, n} \\
a_{i+1\, 1} & a_{i+1\, 2} & \cdots & a_{i+1\, n} \\
\vdots & \vdots & \ddots & \vdots \\
a_{n1} & a_{n2} & \cdots & a_{nn}
\end{vmatrix} + \ldots + 
\begin{vmatrix}
0 & 0 & \cdots & a_{in} \\
a_{11} & a_{12} & \cdots & a_{1n} \\
\vdots & \vdots & \ddots & \vdots \\
a_{i-1\, 1} & a_{i-1\, 2} & \cdots & a_{i-1\, n} \\
a_{i+1\, 1} & a_{i+1\, 2} & \cdots & a_{i+1\, n} \\
\vdots & \vdots & \ddots & \vdots \\
a_{n1} & a_{n2} & \cdots & a_{nn}
\end{vmatrix}
\right) =
\end{equation*}
\begin{equation*}
= (-1)^{i+1} \cdot \left(
\begin{vmatrix}
a_{i1} & 0 & \cdots & 0 \\
a_{11} & a_{12} & \cdots & a_{1n} \\
\vdots & \vdots & \ddots & \vdots \\
a_{i-1\, 1} & a_{i-1\, 2} & \cdots & a_{i-1\, n} \\
a_{i+1\, 1} & a_{i+1\, 2} & \cdots & a_{i+1\, n} \\
\vdots & \vdots & \ddots & \vdots \\
a_{n1} & a_{n2} & \cdots & a_{nn}
\end{vmatrix} -
\begin{vmatrix}
a_{i2} & 0 & 0 & \cdots & 0 \\
a_{12} & a_{11} & a_{13} & \cdots & a_{1n} \\
\vdots & \vdots & \vdots & \ddots & \vdots \\
a_{i-1\, 2} & a_{i-1\, 1} & a_{i-1\, 3} & \cdots & a_{i-1\, n} \\
a_{i+1\, 2} & a_{i+1\, 1} & a_{i+1\, 3} & \cdots & a_{i+1\, n} \\
\vdots & \vdots & \vdots & \ddots & \vdots \\
a_{n2} & a_{n1} & a_{n3} & \cdots & a_{nn}
\end{vmatrix} + \ldots + \right.
\end{equation*}
\begin{equation*}
\left. + (-1)^{n-1} \cdot
\begin{vmatrix}
a_{in} & 0 & \cdots & 0 \\
a_{1n} & a_{11} & \cdots & a_{1\, n-1} \\
\vdots & \vdots & \ddots & \vdots \\
a_{i-1\, n} & a_{i-1\, 1}  & \cdots & a_{i-1\, n-1} \\
a_{i+1\, n} & a_{i+1\, 1} & \cdots & a_{i+1\, n-1} \\
\vdots & \vdots & \ddots & \vdots \\
a_{nn} & a_{n1} & \cdots & a_{n\, n-1}
\end{vmatrix}
\right) =
\end{equation*}
\begin{equation*}
= (-1)^{i+1} a_{i1} \cdot
\begin{vmatrix}
a_{12} & \cdots & a_{1n} \\
\vdots & \ddots & \vdots \\
a_{i-1\, 2} & \cdots & a_{i-1\, n} \\
a_{i+1\, 2} & \cdots & a_{i+1\, n} \\
\vdots & \ddots & \vdots \\
a_{n2} & \cdots & a_{nn}
\end{vmatrix} + (-1)^{i+2} a_{i2} \cdot
\begin{vmatrix}
a_{11} & a_{13} & \cdots & a_{1n} \\
\vdots & \vdots & \ddots & \vdots \\
a_{i-1\, 1} & a_{i-1\, 3} & \cdots & a_{i-1\, n} \\
a_{i+1\, 1} & a_{i+1\, 3} & \cdots & a_{i+1\, n} \\
\vdots & \vdots & \ddots & \vdots \\
a_{n1} & a_{n3} & \cdots & a_{nn}
\end{vmatrix} + \ldots +
\end{equation*}
\begin{equation*}
+ (-1)^{i+n} a_{in} \cdot
\begin{vmatrix}
a_{11} & \cdots & a_{1\, n-1} \\
\vdots & \ddots & \vdots \\
a_{i-1\, 1}  & \cdots & a_{i-1\, n-1} \\
a_{i+1\, 1} & \cdots & a_{i+1\, n-1} \\
\vdots & \ddots & \vdots \\
a_{n1} & \cdots & a_{n\, n-1}
\end{vmatrix}
= \sum_{j=1}^n a_{ij} A_{ij}
\end{equation*}

Аналогично доказывается
\begin{equation*}
\begin{vmatrix}
a_{11} & a_{12} & \cdots & a_{1n} \\
a_{21} & a_{22} & \cdots & a_{2n} \\
\vdots & \vdots & \ddots & \vdots \\
a_{n1} & a_{n2} & \cdots & a_{nn}
\end{vmatrix}
= \sum_{i=1}^n a_{ij} A_{ij}
\end{equation*}
\end{proof}

\textbf{Транспонированием} матрицы или~определителя называется операция, в~результате которой строки меняются местами со~столбцами с~сохранением порядка следования:
\begin{equation*}
\begin{Vmatrix}
a_{11} & a_{12} & \cdots & a_{1n} \\
a_{21} & a_{22} & \cdots & a_{2n} \\
\vdots & \vdots & \ddots & \vdots \\
a_{n1} & a_{n2} & \cdots & a_{nn}
\end{Vmatrix}^T =
\begin{Vmatrix}
a_{11} & a_{21} & \cdots & a_{n1} \\
a_{12} & a_{22} & \cdots & a_{n2} \\
\vdots & \vdots & \ddots & \vdots \\
a_{1n} & a_{2n} & \cdots & a_{nn}
\end{Vmatrix}
\end{equation*}

Полученная матрица или~определитель называется \textbf{транспонированной} по~отношению к~исходной.

\begin{statement}
Определитель транспонированной матрицы равен определителю исходной.
\end{statement}
\begin{proof}
\begin{equation*}
\begin{vmatrix}
a_{11} & a_{12} & \cdots & a_{1n} \\
a_{21} & a_{22} & \cdots & a_{2n} \\
\vdots & \vdots & \ddots & \vdots \\
a_{n1} & a_{n2} & \cdots & a_{nn}
\end{vmatrix}^T =
\begin{vmatrix}
a_{11} & a_{21} & \cdots & a_{n1} \\
a_{12} & a_{22} & \cdots & a_{n2} \\
\vdots & \vdots & \ddots & \vdots \\
a_{1n} & a_{2n} & \cdots & a_{nn}
\end{vmatrix} = \sum_{j=1}^n a_{1j} A_{1j} =
\begin{vmatrix}
a_{11} & a_{12} & \cdots & a_{1n} \\
a_{21} & a_{22} & \cdots & a_{2n} \\
\vdots & \vdots & \ddots & \vdots \\
a_{n1} & a_{n2} & \cdots & a_{nn}
\end{vmatrix}
\end{equation*}
\end{proof}
\subsection{Ранг матрицы}
Строка (столбец) матрицы  называется \textbf{линейно зависимой}, если она является линейной комбинацией остальных строк (столбцов), иначе~--- \textbf{линейно независимой}.

\index{ранг матрицы}\textbf{Рангом} матрицы называется максимальное количество её линейно независимых строк.

\textbf{Минором} $k$\nobreakdash-го порядка матрицы называется определитель, содержащий только те её элементы, которые стоят на~пересечении некоторых $k$~строк и~$k$~столбцов.
Минор наибольшего порядка, отличный от~нуля, называется \textbf{базисным}.

\begin{theorem}
Ранг матрицы равен порядку базисного минора.
\end{theorem}
\begin{proof}
Пусть
\begin{equation*}
A =
\begin{Vmatrix}
a_{11} & a_{12} & \cdots & a_{1n} \\
a_{21} & a_{22} & \cdots & a_{2n} \\
\vdots & \vdots & \ddots & \vdots \\
a_{n1} & a_{n2} & \cdots & a_{nn}
\end{Vmatrix}
\end{equation*}
$M_k$~--- базисный минор $k$\nobreakdash-го порядка.
При перестановке строк и~столбцов минора равенство с~нулём сохраняется, значит, без~ограничения общности можно считать, что
\begin{equation*}
M_k =
\begin{vmatrix}
a_{11} & a_{12} & \cdots & a_{1k} \\
a_{21} & a_{22} & \cdots & a_{2k} \\
\vdots & \vdots & \ddots & \vdots \\
a_{k1} & a_{k2} & \cdots & a_{kk}
\end{vmatrix}
\end{equation*}

$M_k \neq 0$, значит, строки~$A_1, \ldots, A_k$ линейно независимы. Пусть $M_{k+1}$~--- минор $(k + 1)$\nobreakdash-го порядка:
\begin{equation*}
M_{k+1} =
\begin{vmatrix}
a_{11} & a_{12} & \cdots & a_{1k} & a_{1j} \\
a_{21} & a_{22} & \cdots & a_{2k} & a_{2j} \\
\vdots & \vdots & \ddots & \vdots & \vdots \\
a_{k1} & a_{k2} & \cdots & a_{kk} & a_{kj} \\
a_{i1} & a_{i2} & \cdots & a_{ik} & a_{ij}
\end{vmatrix} = 0
\end{equation*}
т.\,к. $M_k$~--- базисный минор.
Тогда
\begin{equation*}
a_{1j} A_{1j} + a_{2j} A_{2j} + \ldots + a_{kj} A_{kj} + a_{ij} A_{ij} = 0, \ A_{ij} = M_k \neq 0 \Rightarrow
\end{equation*}
\begin{equation*}
\Rightarrow a_{ij} = -\frac{A_{1j}}{A_{ij}} a_{1j} - \frac{A_{2j}}{A_{ij}} a_{2j} - \ldots - \frac{A_{kj}}{A_{ij}} a_{kj}
\end{equation*}
где $A_{1j}, \ldots, A_{kj}, A_{ij}$~--- алгебраические дополнения $a_{1j}, \ldots, a_{kj}, a_{ij}$.
$A_{1j}, \ldots, A_{kj}, A_{ij}$ не~зависят от~$j$, тогда $A_i$~--- линейная комбинация $A_1, \ldots, A_k$, значит, $k$~--- ранг матрицы $A$.
\end{proof}

Рангом матрицы по~строкам (столбцам) называется максимальное количество её линейно независимых строк (столбцов).

\begin{consequent}
Ранг матрицы по~строкам равен рангу матрицы по~столбцам.
\end{consequent}%
Для доказательства достаточно заметить, что определитель транспонированной матрицы равен определителю исходной.
% beta
\subsection{Элементарные преобразования матриц}
\textbf{Элементарными преобразованиями} называются следующие операции над~матрицей, не~изменяющие её ранга:
\begin{itemize}
	\item\textbf{Перестановка строк} матриц.
	Очевидно, что ранг матрицы при~перестановке строк не~меняется.
	
	\item\textbf{Умножение строки на~$\lambda \neq 0$.}
	\begin{proof}
	
	\end{proof}
	
	\item\textbf{Прибавление к~строке матрицы другой строки, умноженной на~$\lambda \neq 0$.}
	\begin{proof}
	
	\end{proof}
\end{itemize}

Аналогично определяются элементарные преобразования над столбцами.

Матрица~$A$ имеет \textbf{ступенчатый вид}, если:
\begin{itemize}
	\item все нулевые строки стоят последними;
	\item для любой ненулевой строки~$A_p$ верно, что $\forall i > p, \ j \leqslant q \ a_{ij} = 0$, где $a_{pq}$~--- первый ненулевой элемент строки~$A_p$.
\end{itemize}

\begin{theorem}
Любую матрицу путём элементарных преобразований только над строками можно привести к ступенчатому виду.
\end{theorem}
\begin{proof}

\end{proof}
% beta
\subsection{Обратные матрицы}
Матрица~$B$ называется \textbf{левой обратной} к~квадратной матрице~$A$, если $BA = E$.

Матрица~$C$ называется \textbf{правой обратной} к~квадратной матрице~$A$, если $AC = E$.

Заметим, что обе матрицы $B$ и~$C$~--- квадратные того же порядка, что и~$A$.

\begin{statement}
Если существуют левая и~правая обратные к~$A$ матрицы $B$ и~$C$, то они совпадают.
\end{statement}
\begin{proof}
$B = BE = BAC = EC = C$
\end{proof}

Т.\,о., матрица~$A^{-1}$ называется \textbf{обратной} к~матрице~$A$, если $A^{-1} A = A A^{-1} = E$.

\begin{theorem}
Пусть даны матрицы
\begin{equation*}
A =
\begin{Vmatrix}
a_{11} & a_{12} & \cdots & a_{1n} \\
a_{21} & a_{22} & \cdots & a_{2n} \\
\vdots & \vdots & \ddots & \vdots \\
a_{n1} & a_{n2} & \cdots & a_{nn}
\end{Vmatrix}, \ 
\tilde A =
\begin{Vmatrix}
A_{11} & A_{12} & \cdots & A_{1n} \\
A_{21} & A_{22} & \cdots & A_{2n} \\
\vdots & \vdots & \ddots & \vdots \\
A_{n1} & A_{n2} & \cdots & A_{nn}
\end{Vmatrix}
\end{equation*}
где $A_{ij}$~--- алгебраическое дополнение $a_{ij}$.

Если $|A| \neq 0$, то
\begin{equation*}
A^{-1} = \frac{\tilde A^T}{|A|}
\end{equation*}
\end{theorem}
\begin{proof}

\end{proof}

\begin{theorem}
Пусть дана невырожденная матрица
\begin{equation*}
A =
\begin{Vmatrix}
a_{11} & a_{12} & \cdots & a_{1n} \\
a_{21} & a_{22} & \cdots & a_{2n} \\
\vdots & \vdots & \ddots & \vdots \\
a_{n1} & a_{n2} & \cdots & a_{nn}
\end{Vmatrix}
\end{equation*}

Присоединим к~ней единичную матрицу:
\begin{equation*}
\begin{Vmatrix}
a_{11} & a_{12} & \cdots & a_{1n} & 1 & 0 & \cdots & 0 \\
a_{21} & a_{22} & \cdots & a_{2n} & 0 & 1 & \cdots & 0 \\
\vdots & \vdots & \ddots & \vdots & \vdots & \vdots & \ddots & \vdots \\
a_{n1} & a_{n2} & \cdots & a_{nn} & 0 & 0 & \cdots & 1
\end{Vmatrix}
\end{equation*}
и~с~помощью элементарных преобразований только над~строками полученной матрицы (или только над столбцами) приведём её левую часть к~единичной матрице.
Тогда правая часть будет обратной к~$A$ матрицей.
\end{theorem}
\begin{proof}

\end{proof}