% beta
\subsection{Линейная независимость}
Уравнение системы линейных уравнений называется \textbf{линейно зависимым}, если соответствующая ему строка расширенной матрицы является нетривиальной линейной комбинацией других строк, иначе~--- \textbf{линейно независимым}.

Система линейных уравнений называется \textbf{линейно зависимой}, если существует нетривиальная линейная комбинация строк расширенной матрицы, в~результате которой получается нулевая строка, иначе~--- \textbf{линейно независимой}.

\begin{statement}
Система линейных уравнений линейно зависима $\Leftrightarrow$ одно из~её уравнений линейно зависимо.
\end{statement}
\begin{proof}
\begin{enumerate}
	\item $\Rightarrow$. Пусть система из строк $A_1, \ldots, A_n$ линейно зависима:
	\begin{equation*}
	\sum_{i=1}^n \alpha_i A_i = O, \ \sum_{i=1}^n \alpha_i^2 \neq 0
	\end{equation*}
	где $O$~--- нулевая строка. Без ограничения общности можно считать, что $\alpha_1 \neq 0$, тогда
	\begin{equation*}
	A_1 = -\sum_{i=2}^n \frac{\alpha_i}{\alpha_1} A_i
	\end{equation*}
	
	Значит, $A_1$~--- линейно зависимая строка.
	
	\item $\Leftarrow$. Пусть одна из строк линейно зависима:
	\begin{equation*}
	A_1 = \sum_{i=2}^n \alpha_i A_i \Leftrightarrow
	1 \cdot A_1 - \alpha_2 A_2 - \ldots - \alpha_n A_n = O
	\end{equation*}
	
	Значит, система линейно зависима.
\end{enumerate}
\end{proof}