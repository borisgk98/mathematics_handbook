% beta
\subsection{Фундаментальная система решений}
\begin{statement}
Однородная линейно независимая система уравнений
\begin{equation*}
\begin{cases}
\displaystyle \sum_{i=1}^n a_{1i} x_i = 0 \\
\displaystyle \sum_{i=1}^n a_{2i} x_i = 0 \\
\vdots \\
\displaystyle \sum_{i=1}^n a_{mi} x_i = 0
\end{cases}
\end{equation*}
задаёт векторное пространство размерности~$n - m$.
\end{statement}
\begin{proof}

\end{proof}

\textbf{Фундаментальной системой решений} однородной системы линейных уравнений называется базис множества всех её решений.

\subsubsection{Нахождение фундаментальной системы решений}
Пусть дана однородная линейно независимая система уравнений:
\begin{equation*}
\begin{cases}
\displaystyle \sum_{i=1}^n a_{1i} x_i = 0 \\
\displaystyle \sum_{i=1}^n a_{2i} x_i = 0 \\
\vdots \\
\displaystyle \sum_{i=1}^n a_{mi} x_i = 0 \\
\end{cases} \Leftrightarrow
\begin{cases}
\displaystyle \sum_{i=1}^m a_{1i} x_i = -\sum_{i=m+1}^n a_{1i} x_i \\
\displaystyle \sum_{i=1}^m a_{2i} x_i = -\sum_{i=m+1}^n a_{2i} x_i \\
\vdots \\
\displaystyle \sum_{i=1}^m a_{mi} x_i = -\sum_{i=m+1}^n a_{mi} x_i \\
\end{cases}
\end{equation*}

Пусть 
\begin{equation*}
(x_{11}; x_{21}; \ldots; x_{m1}; 1; 0; \ldots; 0)
\end{equation*}
\begin{equation*}
(x_{12}; x_{22}; \ldots; x_{m2}; 0; 1; \ldots; 0)
\end{equation*}
\begin{equation*}
\vdots
\end{equation*}
\begin{equation*}
(x_{1\, n-m}; x_{2\, n-m}; \ldots; x_{m\, n-m}; 0; 0; \ldots; 1)
\end{equation*}
являются решениями данной системы. 
Тогда они образуют фундаментальную систему решений.
\begin{proof}

\end{proof}

\begin{theorem}
Общее решение неоднородной системы линейных уравнений равно сумме её частного решения и~общего решения соответствующей однородной системы, т.\,е. с~теми~же самыми коэффициентами при~переменных.
\end{theorem}
\begin{proof}

\end{proof}