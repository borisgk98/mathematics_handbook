% beta
\subsection{Элементарные преобразования матриц}
\textbf{Элементарными преобразованиями} называются следующие операции над~матрицей, не~изменяющие её ранга:
\begin{itemize}
	\item\textbf{Перестановка строк} матриц.
	Очевидно, что ранг матрицы при~перестановке строк не~меняется.
	
	\item\textbf{Умножение строки на~$\lambda \neq 0$.}
	\begin{proof}
	
	\end{proof}
	
	\item\textbf{Прибавление к~строке матрицы другой строки, умноженной на~$\lambda \neq 0$.}
	\begin{proof}
	
	\end{proof}
\end{itemize}

Аналогично определяются элементарные преобразования над столбцами.

Матрица~$A$ имеет \textbf{ступенчатый вид}, если:
\begin{itemize}
	\item все нулевые строки стоят последними;
	\item для любой ненулевой строки~$A_p$ верно, что $\forall i > p, \ j \leqslant q \ a_{ij} = 0$, где $a_{pq}$~--- первый ненулевой элемент строки~$A_p$.
\end{itemize}

\begin{theorem}
Любую матрицу путём элементарных преобразований только над строками можно привести к ступенчатому виду.
\end{theorem}
\begin{proof}

\end{proof}