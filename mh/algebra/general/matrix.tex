\section{Алгебра матриц}
\index{матрица}\textbf{Матрицей} над полем $K$ называется прямоугольная таблица из элементов $K$, содержащая $m$~строк и~$n$~столбцов, и~обозначается
\begin{equation*}
A =
\begin{pmatrix}
a_{11} & a_{12} & \cdots & a_{1n} \\
a_{21} & a_{22} & \cdots & a_{2n} \\
\vdots & \vdots & \ddots & \vdots \\
a_{m1} & a_{m2} & \cdots & a_{mn}
\end{pmatrix} =
\begin{Vmatrix}
a_{11} & a_{12} & \cdots & a_{1n} \\
a_{21} & a_{22} & \cdots & a_{2n} \\
\vdots & \vdots & \ddots & \vdots \\
a_{m1} & a_{m2} & \cdots & a_{mn}
\end{Vmatrix}
\end{equation*}

Числа $m$ и~$n$ называются \textbf{порядками} матрицы.

$i$\nobreakdash-я строка матрицы обозначается~$A_i$, $j$\nobreakdash-й столбец~---~$A^j$.

Две матрицы называются \textbf{равными}, если их порядки и~соответствующие элементы совпадают, иначе~--- \textbf{неравными}.

\textbf{Сложение матриц} определено только над~матрицами одинакового размера.
$$
A + B =
\begin{Vmatrix}
a_{11} & a_{12} & \cdots & a_{1n} \\ 
a_{21} & a_{22} & \cdots & a_{2n} \\ 
\vdots & \vdots & \ddots & \vdots \\ 
a_{m1} & a_{m2} & \cdots & a_{mn}
\end{Vmatrix} +
\begin{Vmatrix}
b_{11} & b_{12} & \cdots & b_{1n} \\ 
b_{21} & b_{22} & \cdots & b_{2n} \\ 
\vdots & \vdots & \ddots & \vdots \\ 
b_{m1} & b_{m2} & \cdots & b_{mn}
\end{Vmatrix} =
\begin{Vmatrix}
a_{11} + b_{11} & a_{12} + b_{12} & \cdots & a_{1n} + b_{1n} \\ 
a_{21} + b_{21} & a_{22} + b_{22} & \cdots & a_{2n} + b_{2n} \\ 
\vdots & \vdots & \ddots & \vdots \\ 
a_{m1} + b_{m1} & a_{m2} + b_{m2} & \cdots & a_{mn} + b_{mn}
\end{Vmatrix}
$$

	
\textbf{Умножение матрицы на~число}
\begin{equation*}
\lambda A =
\lambda \cdot
\begin{Vmatrix}
a_{11} & a_{12} & \cdots & a_{1n} \\ 
a_{21} & a_{22} & \cdots & a_{2n} \\ 
\vdots & \vdots & \ddots & \vdots \\ 
a_{m1} & a_{m2} & \cdots & a_{mn}
\end{Vmatrix} =
\begin{Vmatrix}
\lambda a_{11} & \lambda a_{12} & \cdots & \lambda a_{1n} \\ 
\lambda a_{21} & \lambda a_{22} & \cdots & \lambda a_{2n} \\ 
\vdots & \vdots & \ddots & \vdots \\ 
\lambda a_{m1} & \lambda a_{m2} & \cdots & \lambda a_{mn}
\end{Vmatrix}
\end{equation*}

Относительно этих двух операций \textbf{все матрицы размера $m \times n$ образуют векторное пространство}, которое можно обозначить как $K^{m \times n}$.



\textbf{Умножение матриц} $A \cdot B$ определено, только если количество столбцов в~матрице $A$ совпадает с~количеством строк в~матрице $B$. То есть их размеры согласованы.
\begin{equation*}
\begin{Vmatrix}
a_{11} & a_{12} & \cdots & a_{1k} \\ 
a_{21} & a_{22} & \cdots & a_{2k} \\ 
\vdots & \vdots & \ddots & \vdots \\ 
a_{m1} & a_{m2} & \cdots & a_{mk}
\end{Vmatrix} \cdot
\begin{Vmatrix}
b_{11} & b_{12} & \cdots & b_{1n} \\ 
b_{21} & b_{22} & \cdots & b_{2n} \\ 
\vdots & \vdots & \ddots & \vdots \\ 
b_{k1} & b_{k2} & \cdots & b_{kn}
\end{Vmatrix} =
\begin{Vmatrix}
\sum\limits_{i=1}^k a_{1i}b_{i1} & \sum\limits_{i=1}^k a_{1i}b_{i2} & \cdots & \sum\limits_{i=1}^k a_{1i}b_{in} \\
\sum\limits_{i=1}^k a_{2i}b_{i1} & \sum\limits_{i=1}^k a_{2i}b_{i2} & \cdots & \sum\limits_{i=1}^k a_{2i}b_{in} \\
\vdots & \vdots & \ddots & \vdots \\
\sum\limits_{i=1}^k a_{mi}b_{i1} & \sum\limits_{i=1}^k a_{mi}b_{i2} & \cdots & \sum\limits_{i=1}^k a_{mi}b_{in}
\end{Vmatrix}
\end{equation*}

Пример:
$$
\begin{Vmatrix}
	1 & 0 & 2 \\ 
	0 & -1 & 3 \\ 
\end{Vmatrix} 
\begin{Vmatrix}
2 & -1 \\ 
0 & 5 \\ 
1 & 1 \\ 
\end{Vmatrix} = 
\begin{Vmatrix}
1*2 + 0 * 0 + 2 * 1 & 1 * (-1) + 0 * 5 + 2 * 1 \\ 
0 * 2 + (-1) * 0 + 3 * 1 & 0 * (-1) + (-1) * 5 + 3 * 1 \\ 
\end{Vmatrix}  =
\begin{Vmatrix}
4 & 1 \\ 
3 & -2 \\ 
\end{Vmatrix}
$$

Умножение матриц \textbf{ассоциативно}, то есть $A(BC) = (AB)C$, но не \textbf{коммутативно}.

Если порядки матрицы (количество строк и столбцов) $m = n$, то матрица называется \index{квадратная матрица}\textbf{квадратной}, а число~$m = n$~--- её \textbf{порядком}.

\textbf{Главной} называется диагональ квадратной матрицы, состоящая из~элементов $a_{11}, a_{22}, \ldots, a_{nn}$, а \textbf{побочной}~--- состоящая из~элементов $a_{n1}, a_{n-1\, 2}, \ldots, a_{1n}$.

Квадратная матрица называется диагональной