\texttt{}\section{Векторные пространства}
\index{векторное пространство}\index{линейное пространство|see{векторное пространство}}\textbf{Векторным (или линейным) пространством} над полем $K$ называется множество $V = V(K)$ с операциями сложения и умножения на элементы поля $K$, если:
\begin{enumerate}
	\item относительно сложения $V$ есть абелева группа
	\item $\forall a, b \in V \quad \lambda \in K : \lambda(a + b) = \lambda a + \lambda b \qquad$ (дистрибутивность умножения вектора на скаляр относительно сложения векторов)
	\item $\forall a \in V \quad \lambda, \gamma \in K : (\lambda + \gamma)a = \lambda a + \gamma a \qquad$ (дистрибутивность умножения вектора на скаляр относительно сложения скаляров)
	\item $\forall a \in V : 1a = a \qquad$ (унитарность: умножение на нейтральный (по умножению) элемент поля $K$ сохраняет вектор)
	\item  $\forall a \in V \quad \lambda, \gamma \in K : \lambda (\gamma a) = (\lambda \gamma) a$  (ассоциативность умножения на скаляр)
\end{enumerate}

Элементы поля $K$ называются \index{скаляр}\textbf{скалярами}.

Элементы векторного пространства $V$ называются \index{вектор}\textbf{векторами}. Часто над векторами ставиться верхняя черта, но иногда она опускается.

Подмножество $U$ векторного пространства $V$ называется \index{подпространство}\textbf{подпространством}, если:
\begin{enumerate}
	\item $U$ является подгруппой аддитивной группы $V$
	\item $\forall a \in U \quad \lambda \in K \rightarrow \lambda a \in U$
\end{enumerate}

Пусть $K$ - поле и $V$ - векторное пространство над $K$. Если $v_1, v_2 \ldots v_n$ - векторы, а $a_1, a_2 \ldots a_n$ - скаляры, то \index{линейная комбинация}\textbf{линейная комбинация} этих векторов со скалярами а качестве коэффициентов это:
$$
a_1 v_1 + a_2 v_2 + \ldots + a_n v_n
$$ 

Подмножество $v_1, v_2 \ldots v_n$ векторного пространства называется \index{линейная независимость векторов}\textbf{линейно независимым}, если их единственная линейная комбинация со скалярами $a_1, a_2 \ldots a_n$, равная 0 (нулевому вектору), тривиальна ($a_1 = a_2 = \ldots = a_n = 0$).

\index{базис}\textbf{Базис} — упорядоченный (конечный или бесконечный) набор векторов в векторном пространстве, такой, что любой вектор этого пространства может быть единственным образом представлен в виде линейной комбинации векторов из этого набора. Векторы базиса называются базисными векторами.

Максимальное число линейно независимых элементов векторного пространства есть \index{размерность векторного пространства}\textbf{размерность} векторного пространства. У $n$-мерного векторного пространства $V_n$ ранг $\dim V_n = n$.

% TODO proofs

\begin{theorem}[о базисе]
	Любой вектор~$\overline x$ может быть разложен по~базису~$\overline{e_1}, \ldots, \overline{e_n}$ единственным образом.
\end{theorem}
\begin{proof}
	
\end{proof}

\begin{theorem}
	В~векторном пространстве размерности $n$ любые $n$ линейно независимых векторов образуют его базис.
\end{theorem}
\begin{proof}
	
\end{proof}

\begin{theorem}
	Если векторное пространство имеет базис из~$n$~векторов, то его размерность равна $n$.
\end{theorem}
\begin{proof}
	
\end{proof}

\index{линейная оболочка}\textbf{Линейная оболочка} $\mathcal{V}(X)$ подмножества $X$ линейного пространства $V$ — пересечение всех подпространств $V$, содержащих $X$. Иными словами это совокупность всех векторов пространства $V$, которые являются линейными комбинациями векторов из $X$.

Линейная оболочка является подпространством $V$.

Линейная оболочка также называется подпространством, порождённым $X$. Говорят также, что линейная оболочка $\mathcal{V}(X)$ — пространство, натянутое на множество $X$.

Линейная оболочка $\mathcal{V}(X)$ состоит из всевозможных линейных комбинаций различных конечных подсистем элементов из $X$. В частности, если $X$ — конечное множество, то $\mathcal{V}(X)$ состоит из всех линейных комбинаций элементов $X$. Таким образом, нулевой вектор всегда принадлежит линейной оболочке.

Если $X$ — линейно независимое множество, то оно является базисом $\mathcal{V}(X)$ и тем самым определяет его размерность.

\index{ранг}\textbf{Ранг} подмножества $X$ линейного пространства $V$ - это размерность линейной оболочки $\mathcal{V}(X)$.