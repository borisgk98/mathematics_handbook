\texttt{}\section{Векторные пространства}
\index{векторное пространство}\index{линейное пространство|see{векторное пространство}}\textbf{Векторным (или линейным) пространством} над полем $K$ называется множество $V = V(K)$ с операциями сложения и умножения на элементы поля $K$, если:
\begin{enumerate}
	\item относительно сложения $V$ есть абелева группа
	\item $\forall a, b \in V \quad \lambda \in K : \lambda(a + b) = \lambda a + \lambda b \qquad$ (дистрибутивность умножения вектора на скаляр относительно сложения векторов)
	\item $\forall a \in V \quad \lambda, \gamma \in K : (\lambda + \gamma)a = \lambda a + \gamma a \qquad$ (дистрибутивность умножения вектора на скаляр относительно сложения скаляров)
	\item $\forall a \in V : 1a = a \qquad$ (унитарность: умножение на нейтральный (по умножению) элемент поля $K$ сохраняет вектор)
	\item  $\forall a \in V \quad \lambda, \gamma \in K : \lambda (\gamma a) = (\lambda \gamma) a$  (ассоциативность умножения на скаляр)
\end{enumerate}

Элементы поля $K$ называются \index{скаляр}\textbf{скалярами}.

Элементы векторного пространства $V$ называются \index{вектор}\textbf{векторами}.

Пусть $K$ - поле и $V$ - векторное пространство над $K$. Если $v_1, v_2 \ldots v_n$ - векторы, а $a_1, a_2 \ldots a_n$ - скаляры, то \index{линейная комбинация}\textbf{линейная комбинация} этих векторов со скалярами а качестве коэффициентов это:
$$
a_1 v_1 + a_2 v_2 + \ldots + a_n v_n
$$ 

Подмножество $v_1, v_2 \ldots v_n$ векторного пространства называется \index{линейная независимость векторов}\textbf{линейно независимым}, если их единственная линейная комбинация со скалярами $a_1, a_2 \ldots a_n$, равная 0 (нулевому вектору), тривиальна ($a_1 = a_2 = \ldots = a_n = 0$).

\index{базис}\textbf{Базис} — упорядоченный (конечный или бесконечный) набор векторов в векторном пространстве, такой, что любой вектор этого пространства может быть единственным образом представлен в виде линейной комбинации векторов из этого набора. Векторы базиса называются базисными векторами.

Максимальное число линейно независимых элементов векторного пространства есть \index{размерность векторного пространства}\textbf{размерность} (ранг) векторного пространства. У $n$-мерного векторного пространства $V_n$ ранг $\dim V_n = n$.