\section{Кольца}

\index{кольцо}\textbf{Кольцо} - это алгебраическая структура с операциями сложения и умножения $(S, +, *)$, обладающая следующими свойствами:
\begin{enumerate}
	\item Относительно сложения это аддитивная абелева группа.
	
	\item Выполняется \index{дистрибутивность}\textbf{дистрибутивность} умножения относительно сложения, то есть 
	$$
	\forall a, b, c \in S : 
	\begin{cases}
		a*(b+c) = a*b + a*c\\
		(b + c)* a = b * a + c * a\\
	\end{cases}
	$$
	Есть левая и правая дистрибутивность, и если кольцо не коммутативно то они не эквивалентны.
\end{enumerate}

Кольцо называется ассоциативным, если умножение в нём ассоциативно.

Кольцо называется коммутативным, если умножение в нём коммутативно.