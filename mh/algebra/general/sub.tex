\section{Подгруппы, подкольца и подполя}
Пусть $M$ - множество с операцией $\circ$, а $N$ - его подмножество. Тогда говорят, что $N$ \textbf{замкнуто} относительно операции $\circ$, если
$$
\forall a, b \in N : a \circ b \in N
$$

Подмножество сохраняет все свойства множества, задающиеся через тождество, например ассоциативность и коммутативность, но может не сохранять ноль и единицу, а так же существование обратных элементов.

Подмножество $B$ аддитивной абелевой группы называется \index{подгруппа!аддитивной группы}\textbf{подгруппой}, если 
\begin{enumerate}
	\item $B$ замкнуто относительно сложения.
	\item $a \in B \rightarrow -a \in B$
	\item $0 \in B$
\end{enumerate}

Пример: В аддитивной группе $R^+$ имеется следующая цепочка подгрупп:
$$
\mathbb{Z} \subset \mathbb{Q} \subset \mathbb{R}
$$

Подмножество $B$ мультипликативной абелевой группы называется \index{подгруппа!мультипликативной группы}\textbf{подгруппой}, если 
\begin{enumerate}
	\item $B$ замкнуто относительно умножения.
	\item $a \in B \rightarrow a^{-1} \in B$
	\item $1 \in B$
\end{enumerate}

Пример: В мультипликативной группе $R^*$ имеется следующая цепочка подгрупп:
$$
{\pm1} \subset \mathbb{Q} \subset \mathbb{R}
$$

Подмножество $L$ кольца $K$ называется \index{подкольцо}\textbf{подкольцом}, если
\begin{enumerate}
	\item $L$ является подгруппой аддитивной группы кольца $K$
	\item $L$ замкнуто относительно сложения
\end{enumerate}

Подкольцо само является кольцом.

Пример: В аддитивной группе $R^+$ имеется следующая цепочка подколец:
$$
\mathbb{Z} \subset \mathbb{Q} \subset \mathbb{R}
$$

Подмножество $L$ поля $K$ называется \index{подполе}\textbf{подполем}, если
\begin{enumerate}
	\item $L$ является подкольцом кольца $K$
	\item $a \in L, a \neq 0 \rightarrow a^{-1} \in L$
	\item $1 \in L$
\end{enumerate}

Подполе само является полем.

Пример: Поле $\mathbb{Q}$ является подполем $\mathbb{R}$.