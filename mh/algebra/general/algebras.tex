\section{Алгебры}
\index{алгебра}\textbf{Алгеброй} над полем $K$ называют множество $A$ с операцией сложения, умножения и умножения на элементы $K$ (скаляры), обладающее следующими свойствами:
\begin{enumerate}
	\item относительно сложения и умножения на скаляр $A$ - векторное пространство.
	\item относительно сложения и умножения $A$ есть кольцо.
	\item $\forall \lambda \in K \quad a, b \in A : (\lambda a) b = a (\lambda b) = \lambda (ab)$ 
\end{enumerate}

 Например:
 \begin{enumerate}
 	\item Всякое поле $L$, содержащие $K$ в качестве подполя можно рассматривать как алгебру над $K$. В частности $\mathbb{C}$ есть алгебра над $\mathbb{R}$.
 	\item Пространство $E^3$ есть алгебра относительно векторного умножения.
 \end{enumerate}