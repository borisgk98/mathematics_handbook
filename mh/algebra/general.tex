\chapter{Общая алгебра}
\index{алгебраическая структура}\textbf{Алгебраическая структура} - это множество элементов с определёнными на них операциями. Определение алгебраической структуры можно записать как набор значений в скобках.
Например $(M, +, *)$ - алгебраическая структура с определёнными операциями сложения и умножения.

% 1 лекция 
% операторы унарные бинарные тернарные ..
% операции сложения и умножения, противоположные и обратные элементы (нейтральные), аддитивная и мультипликативная запись
% ассоциативность и коммутативность
% инфиксная запись, постфиксная префиксная интерфиксная запись
% 2 лекция
% моноиды, регулярные справа и слева элем, обратимые элементы

Операнд - аргумент операции.

Операция $\circ$ в множестве $M$ - это какое то отображение $M \circ M  \rightarrow M$. В алгебре обычно принято записывать операции в виде инфиксной записи (оператор пишется между операндами, например $a + b$).

Если $\forall a,b,c \in M : (a \circ b) \circ c = a \circ (b \circ c)$, то говорят, что $\circ$ -  \index{ассоциативная операция}\textbf{ассоциативная операция}.

Если $\forall a,b \in M : a \circ b = b \circ a$, то говорят, что $\circ$ - \index{коммутативная операция}\textbf{коммутативная операция}.

Пусть $M$ - множество с операцией $\circ$, а $N$ - множество с операцией $*$. Алгебраические структуры $(M, \circ)$ и $($N$, *)$ называют \textbf{изоморфными}, если существует такое биективное отображение
$$
f: M \rightarrow N
$$
что
$$
f(a \circ b) = f(a) * f(b)
$$
где $a, b \in M$. В этом случае говорят $(M, \circ) \simeq ($N$, *)$. Само отображение $f$ называют \index{изоморфизм!алгебраических структур} \textbf{изоморфизмом}.

\index{нейтральный элемент}\textbf{Нейтральный элемент} множества $M$  с операцией $\circ$ - это такой элемент $e \in M$, что $\forall x \in M : x \circ e = e \circ x = x$.


\index{аддитивная запись}\textbf{Аддитивная запись} - запись через оператор сложения$\mathbf{+}$. $0$ \textbf{(Ноль)} - это нейтральный элемент при аддитивной записи, то есть $x + 0 = 0 + x = 0$. 

Элемент $x'$ для элемента $x$ называется противоположным справа, если $x + x' = 0$.

Элемент $x'$ для элемента $x$ называется противоположным слева, если $x' + x = 0$.

Элемент $x'$ для элемента $x$ называется \index{противоположный элемент}\textbf{противоположным}, если $x + x' = x' + x = 0$. Так же элемент, противоположный $x$ обозначается как $-x$.


\index{мультипликативная запись}\textbf{Мультипликативная запись} - запись через оператор умножения $\mathbf{*}$. $1$ \textbf{(Единица)} - это нейтральный элемент при аддитивной записи, то есть $x * 1 = 1 * x = 1$. 

Элемент $x'$ для элемента $x$ называется обратным справа, если $x * x' = 1$.

Элемент $x'$ для элемента $x$ называется обратным слева, если $x' * x = 0$.

Элемент $x'$ для элемента $x$ называется \index{обратный элемент}\textbf{обратным}, если $x * x' = x' * x = 0$. Так же элемент, обратным $x$ обозначается как $x^{-1}$.

\index{обратимый элемент}\textbf{Обратимый элемент} - это элемент, для которого существует обратный.

\section{Теория групп}
\index{полугруппа}\textbf{Полугруппа} - это множество с заданной на нём ассоциативной бинарной операцией $(S, *)$.

\index{моноид}\textbf{Моноид} - это полугруппа с единицей.

\index{аддитивная абелева группа}\textbf{Аддитивная абелева группа} - это множество с заданной на нём операцией сложения $(S, +)$ и обладающее свойствами коммутативности и ассоциативности, существования нуля (единственного) и существования для каждого элемента (единственного) противоположного ему.

\index{мультипликативная абелева группа}\textbf{Мультипликативная абелева группа} - это множество с заданной на нём операцией умножения $(S, *)$ и обладающее свойствами коммутативности и ассоциативности, существования единицы (единственной) и существования для каждого элемента (единственного) обратного ему.




\section{Теория колец}

\index{кольцо}\textbf{Кольцо} - это алгебраическая структура с операциями сложения и умножения $(S, +, *)$, обладающая следующими свойствами:
\begin{enumerate}
	\item Относительно сложения это аддитивная абелева группа.
	
	\item Выполняется дистрибутивность умножения относительно сложения, то есть $\forall a, b, c \in S : a*(b+c) = a*b + a*c$. 
\end{enumerate}

Кольцо называется ассоциативным, если умножение в нём ассоциативно.

Кольцо называется коммутативным, если умножение в нём коммутативно.

\section{Поля}
\index{поле}\textbf{Полем} называется коммутативное ассоциативное кольцо с единицей, в котором всякий ненулевой элемент обратим.

Примеры полей: $\mathbb{R}$, $\mathbb{Q}$.	
