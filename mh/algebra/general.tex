\chapter{Общая алгебра}
\index{алгебраическая структура}\textbf{Алгебраическая структура} - это множество элементов с определёнными на них операциями. Определение алгебраической структуры можно записать как набор значений в скобках.
Например $(M, +, *)$ - алгебраическая структура с определёнными операциями сложения и умножения.

% 1 лекция 
% операторы унарные бинарные тернарные ..
% операции сложения и умножения, противоположные и обратные элементы (нейтральные), аддитивная и мультипликативная запись
% ассоциативность и коммутативность
% инфиксная запись, постфиксная префиксная интерфиксная запись
% 2 лекция
% моноиды, регулярные справа и слева элем, обратимые элементы

Операнд - аргумент операции.

Операция $\circ$ в множестве $M$ - это какое то отображение $M \circ M  \rightarrow M$. В алгебре обычно принято записывать операции в виде инфиксной записи (оператор пишется между операндами, например $a + b$).

Если $\forall a,b,c \in M : (a \circ b) \circ c = a \circ (b \circ c)$, то говорят, что $\circ$ -  \index{ассоциативная операция}\textbf{ассоциативная операция}.

Если $\forall a,b \in M : a \circ b = b \circ a$, то говорят, что $\circ$ - \index{коммутативная операция}\textbf{коммутативная операция}.

Пусть $M$ - множество с операцией $\circ$, а $N$ - множество с операцией $*$. Алгебраические структуры $(M, \circ)$ и $($N$, *)$ называют \textbf{изоморфными}, если существует такое биективное отображение
$$
f: M \rightarrow N
$$
что
$$
f(a \circ b) = f(a) * f(b)
$$
где $a, b \in M$. В этом случае говорят $(M, \circ) \simeq ($N$, *)$. Само отображение $f$ называют \index{изоморфизм!алгебраических структур} \textbf{изоморфизмом}.

\index{нейтральный элемент}\textbf{Нейтральный элемент} множества $M$  с операцией $\circ$ - это такой элемент $e \in M$, что $\forall x \in M : x \circ e = e \circ x = x$.


\index{аддитивная запись}\textbf{Аддитивная запись} - запись через оператор сложения$\mathbf{+}$. $0$ \textbf{(Ноль)} - это нейтральный элемент при аддитивной записи, то есть $x + 0 = 0 + x = 0$. 

Элемент $x'$ для элемента $x$ называется противоположным справа, если $x + x' = 0$.

Элемент $x'$ для элемента $x$ называется противоположным слева, если $x' + x = 0$.

Элемент $x'$ для элемента $x$ называется \index{противоположный элемент}\textbf{противоположным}, если $x + x' = x' + x = 0$. Так же элемент, противоположный $x$ обозначается как $-x$.


\index{мультипликативная запись}\textbf{Мультипликативная запись} - запись через оператор умножения $\mathbf{*}$. $1$ \textbf{(Единица)} - это нейтральный элемент при аддитивной записи, то есть $x * 1 = 1 * x = 1$. 

Операция умножения зачастую не пишется, то есть $a * b = ab$.

Элемент $x'$ для элемента $x$ называется обратным справа, если $x * x' = 1$.

Элемент $x'$ для элемента $x$ называется обратным слева, если $x' * x = 0$.

Элемент $x'$ для элемента $x$ называется \index{обратный элемент}\textbf{обратным}, если $x * x' = x' * x = 0$. Так же элемент, обратным $x$ обозначается как $x^{-1}$.

\index{обратимый элемент}\textbf{Обратимый элемент} - это элемент, для которого существует обратный.

\section{Группы}
\index{полугруппа}\textbf{Полугруппа} - это множество с заданной на нём ассоциативной бинарной операцией $(S, *)$.

\index{моноид}\textbf{Моноид} - это полугруппа с единицей.

\textbf{Группа} - это такая алгебраическая структура $(S, *, -1, 1)$, для которой выполняется три свойства:
\begin{enumerate}
	\item $*$ задаёт операцию на множестве $S$.
	\item Существование обратного элемента $\forall x \in S \quad \exists x^{-1}$
	\item Существование единицы.
\end{enumerate} 

\index{аддитивная абелева группа}\textbf{Аддитивная абелева группа} - это множество с заданной на нём операцией сложения $(S, +)$ и обладающее свойствами коммутативности и ассоциативности, существования нуля (единственного) и существования для каждого элемента (единственного) противоположного ему.

\index{мультипликативная абелева группа}\textbf{Мультипликативная абелева группа} - это множество с заданной на нём операцией умножения $(S, *)$ и обладающее свойствами коммутативности и ассоциативности, существования единицы (единственной) и существования для каждого элемента (единственного) обратного ему.




\section{Теория колец}

\index{кольцо}\textbf{Кольцо} - это алгебраическая структура с операциями сложения и умножения $(S, +, *)$, обладающая следующими свойствами:
\begin{enumerate}
	\item Относительно сложения это аддитивная абелева группа.
	
	\item Выполняется дистрибутивность умножения относительно сложения, то есть $\forall a, b, c \in S : a*(b+c) = a*b + a*c$. 
\end{enumerate}

Кольцо называется ассоциативным, если умножение в нём ассоциативно.

Кольцо называется коммутативным, если умножение в нём коммутативно.

\section{Поля}
\index{поле}\textbf{Полем} называется коммутативное ассоциативное кольцо с единицей, в котором всякий ненулевой элемент обратим.

Примеры полей: $\mathbb{R}$, $\mathbb{Q}$.	

\section{Подгруппы, подкольца и подполя}
Пусть $M$ - множество с операцией $\circ$, а $N$ - его подмножество. Тогда говорят, что $N$ \textbf{замкнуто} относительно операции $\circ$, если
$$
\forall a, b \in N : a \circ b \in N
$$

Подмножество сохраняет все свойства множества, задающиеся через тождество, например ассоциативность и коммутативность, но может не сохранять ноль и единицу, а так же существование обратных элементов.

Подмножество $B$ аддитивной абелевой группы называется \index{подгруппа!аддитивной группы}\textbf{подгруппой}, если 
\begin{enumerate}
	\item $B$ замкнуто относительно сложения.
	\item $a \in B \rightarrow -a \in B$
	\item $0 \in B$
\end{enumerate}

Пример: В аддитивной группе $R^+$ имеется следующая цепочка подгрупп:
$$
\mathbb{Z} \subset \mathbb{Q} \subset \mathbb{R}
$$

Подмножество $B$ мультипликативной абелевой группы называется \index{подгруппа!мультипликативной группы}\textbf{подгруппой}, если 
\begin{enumerate}
	\item $B$ замкнуто относительно умножения.
	\item $a \in B \rightarrow a^{-1} \in B$
	\item $1 \in B$
\end{enumerate}

Пример: В мультипликативной группе $R^*$ имеется следующая цепочка подгрупп:
$$
{\pm1} \subset \mathbb{Q} \subset \mathbb{R}
$$

Подмножество $L$ кольца $K$ называется \index{подкольцо}\textbf{подкольцом}, если
\begin{enumerate}
	\item $L$ является подгруппой аддитивной группы кольца $K$
	\item $L$ замкнуто относительно сложения
\end{enumerate}

Подкольцо само является кольцом.

Пример: В аддитивной группе $R^+$ имеется следующая цепочка подколец:
$$
\mathbb{Z} \subset \mathbb{Q} \subset \mathbb{R}
$$

Подмножество $L$ поля $K$ называется \index{подполе}\textbf{подполем}, если
\begin{enumerate}
	\item $L$ является подкольцом кольца $K$
	\item $a \in L, a \neq 0 \rightarrow a^{-1} \in L$
	\item $1 \in L$
\end{enumerate}

Подполе само является полем.

Пример: Поле $\mathbb{Q}$ является подполем $\mathbb{R}$. % подгруппы подкольца и подполя

\section{Поле комплексных чисел}
\index{поле!комплексных чисел}\textbf{Поле комплексных чисел} - это всякое поле $\mathbb{C}$, обладающее следующими свойствами:
\begin{enumerate}
	\item оно содержит в качестве подполя поле $\mathbb{R}$ вещественных чисел.
	\item Оно содержит такой элемент $i$, что $i^2 = -1$
	\item Оно минимально среди всех полей, обладающих таким свойством, то если $K \subset \mathbb{C}$ какое-либо подполе, содержащие и $\mathbb{R}$ и $i$, то $K = \mathbb{C}$
\end{enumerate}

\begin{theorem}
	Поле комплексных чисел существует и единственно с точностью до изоморфизма, переводящего все вещественные числа в себя. Каждое комплексное число представимо в виде $a + bi$, где $a, b \in \mathbb{R}$, $i$ - (фиксированный) элемент, квадрат которого равен $1$.
\end{theorem}

% TODO доказательство



\texttt{}\section{Векторные пространства}
\index{векторное пространство}\index{линейное пространство|see{векторное пространство}}\textbf{Векторным (или линейным) пространством} над полем $K$ называется множество $V = V(K)$ с операциями сложения и умножения на элементы поля $K$, если:
\begin{enumerate}
	\item относительно сложения $V$ есть абелева группа
	\item $\forall a, b \in V \quad \lambda \in K : \lambda(a + b) = \lambda a + \lambda b \qquad$ (дистрибутивность умножения вектора на скаляр относительно сложения векторов)
	\item $\forall a \in V \quad \lambda, \gamma \in K : (\lambda + \gamma)a = \lambda a + \gamma a \qquad$ (дистрибутивность умножения вектора на скаляр относительно сложения скаляров)
	\item $\forall a \in V : 1a = a \qquad$ (унитарность: умножение на нейтральный (по умножению) элемент поля $K$ сохраняет вектор)
	\item  $\forall a \in V \quad \lambda, \gamma \in K : \lambda (\gamma a) = (\lambda \gamma) a$  (ассоциативность умножения на скаляр)
\end{enumerate}

Элементы поля $K$ называются \index{скаляр}\textbf{скалярами}.

Элементы векторного пространства $V$ называются \index{вектор}\textbf{векторами}.

Пусть $K$ - поле и $V$ - векторное пространство над $K$. Если $v_1, v_2 \ldots v_n$ - векторы, а $a_1, a_2 \ldots a_n$ - скаляры, то \index{линейная комбинация}\textbf{линейная комбинация} этих векторов со скалярами а качестве коэффициентов это:
$$
a_1 v_1 + a_2 v_2 + \ldots + a_n v_n
$$ 

Подмножество $v_1, v_2 \ldots v_n$ векторного пространства называется \index{линейная независимость векторов}\textbf{линейно независимым}, если их единственная линейная комбинация со скалярами $a_1, a_2 \ldots a_n$, равная 0 (нулевому вектору), тривиальна ($a_1 = a_2 = \ldots = a_n = 0$).

\index{базис}\textbf{Базис} — упорядоченный (конечный или бесконечный) набор векторов в векторном пространстве, такой, что любой вектор этого пространства может быть единственным образом представлен в виде линейной комбинации векторов из этого набора. Векторы базиса называются базисными векторами.

Максимальное число линейно независимых элементов векторного пространства есть \index{размерность векторного пространства}\textbf{размерность} (ранг) векторного пространства. У $n$-мерного векторного пространства $V_n$ ранг $\dim V_n = n$.

\section{Алгебры}
\index{алгебра}\textbf{Алгеброй} над полем $K$ называют множество $A$ с операцией сложения, умножения и умножения на элементы $K$ (скаляры), обладающее следующими свойствами:
\begin{enumerate}
	\item относительно сложения и умножения на скаляр $A$ - векторное пространство.
	\item относительно сложения и умножения $A$ есть кольцо.
	\item $\forall \lambda \in K \quad a, b \in A : (\lambda a) b = a (\lambda b) = \lambda (ab)$ 
\end{enumerate}

 Например:
 \begin{enumerate}
 	\item Всякое поле $L$, содержащие $K$ в качестве подполя можно рассматривать как алгебру над $K$. В частности $\mathbb{C}$ есть алгебра над $\mathbb{R}$.
 	\item Пространство $E^3$ есть алгебра относительно векторного умножения.
 \end{enumerate}

\section{Алгебра матриц}
\index{матрица}\textbf{Матрицей} над полем $K$ называется прямоугольная таблица из элементов $K$, содержащая $m$~строк и~$n$~столбцов, и~обозначается
\begin{equation*}
A =
\begin{pmatrix}
a_{11} & a_{12} & \cdots & a_{1n} \\
a_{21} & a_{22} & \cdots & a_{2n} \\
\vdots & \vdots & \ddots & \vdots \\
a_{m1} & a_{m2} & \cdots & a_{mn}
\end{pmatrix} =
\begin{Vmatrix}
a_{11} & a_{12} & \cdots & a_{1n} \\
a_{21} & a_{22} & \cdots & a_{2n} \\
\vdots & \vdots & \ddots & \vdots \\
a_{m1} & a_{m2} & \cdots & a_{mn}
\end{Vmatrix}
\end{equation*}

Числа $m$ и~$n$ называются \textbf{порядками} матрицы.

$i$\nobreakdash-я строка матрицы обозначается~$A_i$, $j$\nobreakdash-й столбец~---~$A^j$.

Две матрицы называются \textbf{равными}, если их порядки и~соответствующие элементы совпадают, иначе~--- \textbf{неравными}.

\textbf{Сложение матриц} определено только над~матрицами одинакового размера.
$$
A + B =
\begin{Vmatrix}
a_{11} & a_{12} & \cdots & a_{1n} \\ 
a_{21} & a_{22} & \cdots & a_{2n} \\ 
\vdots & \vdots & \ddots & \vdots \\ 
a_{m1} & a_{m2} & \cdots & a_{mn}
\end{Vmatrix} +
\begin{Vmatrix}
b_{11} & b_{12} & \cdots & b_{1n} \\ 
b_{21} & b_{22} & \cdots & b_{2n} \\ 
\vdots & \vdots & \ddots & \vdots \\ 
b_{m1} & b_{m2} & \cdots & b_{mn}
\end{Vmatrix} =
\begin{Vmatrix}
a_{11} + b_{11} & a_{12} + b_{12} & \cdots & a_{1n} + b_{1n} \\ 
a_{21} + b_{21} & a_{22} + b_{22} & \cdots & a_{2n} + b_{2n} \\ 
\vdots & \vdots & \ddots & \vdots \\ 
a_{m1} + b_{m1} & a_{m2} + b_{m2} & \cdots & a_{mn} + b_{mn}
\end{Vmatrix}
$$

	
\textbf{Умножение матрицы на~число}
\begin{equation*}
\lambda A =
\lambda \cdot
\begin{Vmatrix}
a_{11} & a_{12} & \cdots & a_{1n} \\ 
a_{21} & a_{22} & \cdots & a_{2n} \\ 
\vdots & \vdots & \ddots & \vdots \\ 
a_{m1} & a_{m2} & \cdots & a_{mn}
\end{Vmatrix} =
\begin{Vmatrix}
\lambda a_{11} & \lambda a_{12} & \cdots & \lambda a_{1n} \\ 
\lambda a_{21} & \lambda a_{22} & \cdots & \lambda a_{2n} \\ 
\vdots & \vdots & \ddots & \vdots \\ 
\lambda a_{m1} & \lambda a_{m2} & \cdots & \lambda a_{mn}
\end{Vmatrix}
\end{equation*}

Относительно этих двух операций \textbf{все матрицы размера $m \times n$ образуют векторное пространство}, которое можно обозначить как $K^{m \times n}$.



\textbf{Умножение матриц} $A \cdot B$ определено, только если количество столбцов в~матрице $A$ совпадает с~количеством строк в~матрице $B$. То есть их размеры согласованы.
\begin{equation*}
\begin{Vmatrix}
a_{11} & a_{12} & \cdots & a_{1k} \\ 
a_{21} & a_{22} & \cdots & a_{2k} \\ 
\vdots & \vdots & \ddots & \vdots \\ 
a_{m1} & a_{m2} & \cdots & a_{mk}
\end{Vmatrix} \cdot
\begin{Vmatrix}
b_{11} & b_{12} & \cdots & b_{1n} \\ 
b_{21} & b_{22} & \cdots & b_{2n} \\ 
\vdots & \vdots & \ddots & \vdots \\ 
b_{k1} & b_{k2} & \cdots & b_{kn}
\end{Vmatrix} =
\begin{Vmatrix}
\sum\limits_{i=1}^k a_{1i}b_{i1} & \sum\limits_{i=1}^k a_{1i}b_{i2} & \cdots & \sum\limits_{i=1}^k a_{1i}b_{in} \\
\sum\limits_{i=1}^k a_{2i}b_{i1} & \sum\limits_{i=1}^k a_{2i}b_{i2} & \cdots & \sum\limits_{i=1}^k a_{2i}b_{in} \\
\vdots & \vdots & \ddots & \vdots \\
\sum\limits_{i=1}^k a_{mi}b_{i1} & \sum\limits_{i=1}^k a_{mi}b_{i2} & \cdots & \sum\limits_{i=1}^k a_{mi}b_{in}
\end{Vmatrix}
\end{equation*}

Пример:
$$
\begin{Vmatrix}
	1 & 0 & 2 \\ 
	0 & -1 & 3 \\ 
\end{Vmatrix} 
\begin{Vmatrix}
2 & -1 \\ 
0 & 5 \\ 
1 & 1 \\ 
\end{Vmatrix} = 
\begin{Vmatrix}
1*2 + 0 * 0 + 2 * 1 & 1 * (-1) + 0 * 5 + 2 * 1 \\ 
0 * 2 + (-1) * 0 + 3 * 1 & 0 * (-1) + (-1) * 5 + 3 * 1 \\ 
\end{Vmatrix}  =
\begin{Vmatrix}
4 & 1 \\ 
3 & -2 \\ 
\end{Vmatrix}
$$

Умножение матриц \textbf{ассоциативно}, то есть $A(BC) = (AB)C$, но не \textbf{коммутативно}.

Если порядки матрицы (количество строк и столбцов) $m = n$, то матрица называется \index{квадратная матрица}\textbf{квадратной}, а число~$m = n$~--- её \textbf{порядком}.

\textbf{Главной} называется диагональ квадратной матрицы, состоящая из~элементов $a_{11}, a_{22}, \ldots, a_{nn}$, а \textbf{побочной}~--- состоящая из~элементов $a_{n1}, a_{n-1\, 2}, \ldots, a_{1n}$.

Квадратная матрица называется диагональной

