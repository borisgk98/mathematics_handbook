\hypertarget{pigeonhole_principle}{\section{Принцип Дирихле}}
В комбинаторике \textbf{принцип Дирихле} — утверждение, сформулированное немецким математиком Дирихле в 1834 году, устанавливающее связь между объектами («кроликами») и контейнерами («клетками») при выполнении определённых условий. 

Наиболее тривиальная формулировка принципа Дирихле звучит так:
\begin{center}
\textit{Если кролики рассажены в клетки, причём число кроликов больше числа клеток, то хотя бы в одной из клеток находится более одного кролика.}
\end{center}