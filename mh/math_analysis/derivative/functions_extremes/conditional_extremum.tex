\subsection{Экстремум с условием. Метод множителей Лагранжа}
Пусть дана функция $f(x_1 \ldots x_n)$ и несколько условий $u_1(x_1 \ldots x_n) = 0 \ldots u_k(x_1 \ldots x_n) = 0$. Нужно найти экстремум функции при этих условиях. Метод множителей Лагранжа:
\begin{enumerate}
	\item Составим \textbf{функцию Лагранжа} от $n + k$ переменных $L(x_1 \ldots x_n, \lambda_1 \ldots \lambda_k) = f(x_1 \ldots x_n) + \sum\limits_{i = 1}^{k} \lambda_i u_i(x_1 \ldots x_n)$.
	\item Составим систему уравнений, приравняв частные производные $L$ к $0$.
	\item Если полученная система имеет решение относительно параметров $x'_{j}$ и $\lambda '_{i}$, тогда точка $x'$ может быть условным экстремумом, то есть решением исходной задачи. Заметим, что это условие носит необходимый, но не достаточный характер.\\
	Проверка точки для функции двух переменных: найдём дифференциал второго порядка $d^2 L = L''_{xx} (dx)^2 + 2L''_{xy}dxdy + L''_{yy}$. Если $d^2 L > 0 \quad \forall x, y$ то функция достигает минимума в точке $x'$, если $d^2 L < 0 \quad \forall x, y$, то функция достигает максимума в точке $x'$.
\end{enumerate} 