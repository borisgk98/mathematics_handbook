\subsection{Экстремумы функции трёх переменных}
Для того, чтобы найти экстремум функции $f(x,y,z)$ двух переменных, нужно найти точки, в которых частные производные 1-ого порядка равны $0$. Пусть мы нашли такую точку $M_0(x_0,y_0,z_0)$. Тогда найдём производные второго порядка в этой точке, вычислим их и составим \textbf{матрицу Гессе}:
$$
\begin{pmatrix}
f''_{xx}(M_0) & f''_{xy}(M_0) & f''_{xz}(M_0)\\
f''_{yx}(M_0) & f''_{yy}(M_0) & f''_{yz}(M_0)\\
f''_{zx}(M_0) & f''_{zy}(M_0) & f''_{zz}(M_0)\\
\end{pmatrix}
$$
Найдём \textbf{угловые миноры}:
$\sigma_1 = f''_{xx}(M_0)$,
$\sigma_2 = 
\begin{bmatrix}
f''_{xx}(M_0) & f''_{xy}(M_0)\\
f''_{yx}(M_0) & f''_{yy}(M_0)
\end{bmatrix}$,
$
\sigma_3 = 
\begin{bmatrix}
f''_{xx}(M_0) & f''_{xy}(M_0) & f''_{xz}(M_0)\\
f''_{yx}(M_0) & f''_{yy}(M_0) & f''_{yz}(M_0)\\
f''_{zx}(M_0) & f''_{zy}(M_0) & f''_{zz}(M_0)\\
\end{bmatrix}$\\
Теперь возможны 4 случая:
\begin{enumerate}
	\item Если $\sigma_1 > 0$, $\sigma_2 > 0$ и $\sigma_3 > 0$, то $M_0(x_0,y_0,z_0)$ - точка минимума.	
	\item Если $\sigma_1 < 0$, $\sigma_2 > 0$ и $\sigma_3 < 0$, то $M_0(x_0,y_0,z_0)$ - точка максимума.	
	\item Иначе если $\sigma_3 \neq 0$, то $M_0(x_0,y_0,z_0)$ - седловая точка.	
	\item При $\sigma_3 = 0$, то нужно дополнительное исследование.	
\end{enumerate}
