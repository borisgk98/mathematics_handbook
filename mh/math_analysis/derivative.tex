\chapter{Производная}
\textbf{Определение:}\\
\textbf{Производной функции в точке называется предел отношения приращения функции к приращению аргумента, когда приращение аргумента стремится к 0.}\\
\textbf{\fbox{\parbox{15cm}{
			$$
			\mathbf{
			f'(x_{0}) = \lim_{\triangle x \to 0}{\frac{\triangle f}{\triangle x}} = 
			\lim_{\triangle x \to 0}{\frac{f(x_0 + \triangle x) - f(x_0)}{\triangle x}}
	     	}
			$$
			}}}\\\\
$\triangle x \qquad$ - приращение аргумента, то есть изменение аргумента от $x$ до $x_0$ (дельта $x$).\\
$\triangle f = f(x + \triangle x) - f(x) \qquad$ - приращение функции (дельта $f$).\\

\section{Свойства производных}
\textbf{
	\begin{enumerate}
		\item $ \mathbf{(C * x)' = C * (x)' \qquad C = const } $
		\item $ \mathbf{(f + g)' = f' + g' }$
		\item $ \mathbf{(f * g)' = f' * g + g' * f }$
		\item $ \mathbf{ \left(\dfrac{f}{g}\right)' = \dfrac{f' * g - g' * f}{g^2} } $
		\item $ \mathbf{ (f(g))' = f'(g) * g'(f) } $
		\item $ \mathbf{ (f^g) = f^g * \ln{f} * g' + g * f^(g - 1) * f' } \qquad $
		\item $ \mathbf{ f'(y) = \dfrac{1}{g(x)} } \qquad $
		 $\mathbf{f(y)}$ и $\mathbf{g(x)}$ - взаимообратные функции ($\mathbf{D(f(y)) = E(g(x))}$ и $\mathbf{D(g(x)) = E(f(y))}$).
	\end{enumerate}
}

\subsection{Экстремумы функции двух переменных}
Для того, чтобы найти экстремум функции $z(x,y)$ двух переменных, нужно найти точки, в которых частные производные 1-ого порядка равны $0$. Пусть мы нашли такую точку $M_0(x_0,y_0)$. Тогда найдём производные второго порядка в этой точке $A = z''_{xx}(x_0,y_0)$, $B = z''_{xy}(x_0,y_0)$ и $C = z''_{yy}(x_0,y_0)$. Если $AC - B^2 > 0$, то $z(x,y)$ имеет экстремум в точке $M_0$ (если $A > 0$, то минимум, если $A < 0$, то максимум).
\subsection{Экстремумы функции трёх переменных}
Для того, чтобы найти экстремум функции $f(x,y,z)$ двух переменных, нужно найти точки, в которых частные производные 1-ого порядка равны $0$. Пусть мы нашли такую точку $M_0(x_0,y_0,z_0)$. Тогда найдём производные второго порядка в этой точке, вычислим их и составим \textbf{матрицу Гессе}:
$$
\begin{pmatrix}
f''_{xx}(M_0) & f''_{xy}(M_0) & f''_{xz}(M_0)\\
f''_{yx}(M_0) & f''_{yy}(M_0) & f''_{yz}(M_0)\\
f''_{zx}(M_0) & f''_{zy}(M_0) & f''_{zz}(M_0)\\
\end{pmatrix}
$$
Найдём \textbf{угловые миноры}:
$\sigma_1 = f''_{xx}(M_0)$,\\
$\sigma_2 = 
\begin{bmatrix}
f''_{xx}(M_0) & f''_{xy}(M_0)\\
f''_{yx}(M_0) & f''_{yy}(M_0)
\end{bmatrix}$,\\
$
\sigma_3 = 
\begin{bmatrix}
f''_{xx}(M_0) & f''_{xy}(M_0) & f''_{xz}(M_0)\\
f''_{yx}(M_0) & f''_{yy}(M_0) & f''_{yz}(M_0)\\
f''_{zx}(M_0) & f''_{zy}(M_0) & f''_{zz}(M_0)\\
\end{bmatrix}$\\
Теперь возможны 4 случая:
\begin{enumerate}
	\item Если $\sigma_1 > 0$, $\sigma_2 > 0$ и $\sigma_3 > 0$, то $M_0(x_0,y_0,z_0)$ - точка минимума.	
	\item Если $\sigma_1 < 0$, $\sigma_2 > 0$ и $\sigma_3 < 0$, то $M_0(x_0,y_0,z_0)$ - точка максимума.	
	\item Иначе если $\sigma_3 \neq 0$, то $M_0(x_0,y_0,z_0)$ - седловая точка.	
	\item При $\sigma_3 = 0$, то нужно дополнительное исследование.	
\end{enumerate}

\subsection{Экстремум с условием. Метод множителей Лагранжа}
Пусть дана функция $f(x_1 \ldots x_n)$ и несколько условий $u_1(x_1 \ldots x_n) = 0 \ldots u_k(x_1 \ldots x_n) = 0$. Нужно найти экстремум функции при этих условиях. Метод множителей Лагранжа:
\begin{enumerate}
	\item Составим \textbf{функцию Лагранжа} от $n + k$ переменных $L(x_1 \ldots x_n, \lambda_1 \ldots \lambda_k) = f(x_1 \ldots x_n) + \sum\limits_{i = 1}^{k} \lambda_i u_i(x_1 \ldots x_n)$.
	\item Составим систему уравнений, приравняв частные производные $L$ к $0$.
	\item Если полученная система имеет решение относительно параметров $x'_{j}$ и $\lambda '_{i}$, тогда точка $x'$ может быть условным экстремумом, то есть решением исходной задачи. Заметим, что это условие носит необходимый, но не достаточный характер.\\
	Проверка точки для функции двух переменных: найдём дифференциал второго порядка $d^2 L = L''_{xx} (dx)^2 + 2L''_{xy}dxdy + L''_{yy}$. Если $d^2 L > 0 \quad \forall x, y$ то функция достигает минимума в точке $x'$, если $d^2 L < 0 \quad \forall x, y$, то функция достигает максимума в точке $x'$.
\end{enumerate} 



\section{Геометрическая интерпретация производной}
\subsection{Касательная}
\subsubsection{В трёхмерном пространстве}
Пусть дана функция, задающая поверхность $F(x,y,z) = 0$.\\
\textbf{Касательная плоскость} к поверхности в точке $M_0$ – это плоскость, содержащая касательные ко всем кривым, которые принадлежат данной поверхности и проходят через точку $M_0$. Её уравнение имеет вид $F'_x(M_0)(x - x_0) + F'_y(M_0)(y - y_0) + F'_z(M_0)(z - z_0) = 0$.
\subsection{Нормаль}
\subsubsection{В трёхмерном пространстве}
Пусть дана функция, задающая поверхность $F(x,y,z) = 0$.\\
\textbf{Нормаль} к поверхности в точке $M_0$ – это прямая, проходящая через данную точку перпендикулярно касательной плоскости. Её каноническое уравнение имеет вид $\frac{x - x_0}{F'_x(M_0)} = \frac{y - y_0}{F'_y(M_0)} = \frac{z - z_0}{F'_z(M_0)}$.