\section{Периодичность ф-ий}
\textbf{Определение:}\\
\textbf{\fbox{\parbox{15cm}{Функцию $y = f(x)$ с областью определения $X$ называют периодической, если 
	$\exists T \neq 0 \quad \forall x \in X$ такой, что $(x + T) \in X$, и $(x - t) \in X$, и $f(x + T) = f(x)$}}}\\

\textbf{Пример ур-ия, где используется периодичность ф-ий:}\\
Пусть $f(x)$ - периодическая функция с периодом 8, такая, что $f(x) = 8x - x^{2}$ при $x \in [0; 8)$. Решите уравнение $f(2x + 16) + 23 = 5f(x)$.\\
Решение:\\
\begin{enumerate}
	\item$$\begin{cases}
				f(x) = f(x + T) = f(x - T)\\
				T = 8\\
			\end{cases}
		 \quad \Longrightarrow \quad f(2x + 16) = f(2x)$$
	\item $x \in [0; 4) \quad \Longrightarrow \quad 2x \in [0; 8)$\\
		Решаем уравнение для этого случая:\\
		$f(2x) + 23 = 5f(x)$\\
		$16x - 4x^{2} + 23 = 40x - 5x^{2}$\\
		$x^{2} - 24x + 23 = 0$\\
		$x1 = 1$\\
		$x2 = 23\qquad \mbox{побочный корень для }x \in [0; 4)$\\
	\item $x \in [4; 8) \quad \Longrightarrow \quad (2x - 8) \in [0; 8)$\\
	Решаем уравнение для этого случая:\\
	$f(2x - 8) + 23 = 5f(x)$\\
	$16x - 64 - 4x^{2} + 16x - 64  + 23 = 40x - 5x^{2}$\\
	$x^{2} - 8x - 105 = 0$\\
	$x1 = 7$\\
	$x2 = -15\qquad \mbox{побочный корень для }x \in [4; 8)$\\
			
	\item Так как наша функция имеет период 8, то и корни будут повторятся с такой же периодичностью, так как $f(x) = f(x + T) = f(x - T)$.
	 То есть получаем корни $x = 1 + 8n$ и $x = 7 + 8n$.
\end{enumerate}
Ответ: $x = 1 + 8n$ и $x = 7 + 8n$.