\part{Теория множеств}             
\chapter{Основные понятия}
\section{Определение}
\index{множество}\textbf{Множество} - ключевое понятие теории множест. Оно аксиоматично, то есть неопределяемо. Обозначаются множества обычно заглавными буквами латинского алфивита.\\
 $\in$ - символ принадлежности множеству.\\
\textbf{Пустое множество} — множество, не содержащее ни одного элемента.\\
\textbf{Универсальное множество (универсум)} — множество, содержащее все мыслимые объекты. В связи с парадоксом Рассела данное понятие трактуется в настоящее время как «множество, включающее все множества, участвующие в рассматриваемой задаче».
\section{Аксиоматика}

\section{Операции на множествами}
\begin{itemize}
	\item Объеденение $A \cup B = C \quad \Leftrightarrow \quad \forall c \in C : c \in A \text{ или } c \in B$
	\item Пересечение $A \cap B = C \quad \Leftrightarrow \quad \forall c \in C : c \in A \text{ и } c \in B$
	\item Разница $A \backslash B = C \quad \Leftrightarrow \quad \forall c \in C : c \in A \text{ и } c \notin B$
	\item Симметрическая разница $A \vartriangle B = C \quad \Leftrightarrow \quad \forall c \in C : c \in A \cup \text{ и } c \notin A \cap B $
	\item Дополнение к множеству $A$ (в универсальном множестве  $M$) $\overline{A} \quad \Leftrightarrow \quad \forall x \in \overline{A}, x \in M : x \notin A$
\end{itemize}

          
\chapter{Функции над множествами}
\section{Определение}
\index{функция}\textbf{Функцией $\mathbf{f : A \rightarrow B}$} называется правило, ставящие в соответствие каждому элементу множества $A$ единственный элемент множества $B$ ($f(a) \in B$, $a \in A$).\\
Множество $A$ - \textbf{область определения $f$}.\\
Множество $B$ - \textbf{область заначения $f$}.\\
\section{Биекции, Иньекции, Сюрьекции}
\begin{itemize}
	\item $f : A \rightarrow B$ называют \textbf{иньективной}, если $\forall x, y \in A : x \neq y \Rightarrow f(x) \neq f(y)$
	\item $f : A \rightarrow B$ называют \textbf{сюрьективной}, если $\forall b \in B \exists a \in A : f(a) = b$
	\item $f : A \rightarrow B$ называют \textbf{биньюктивной}, если она является и иньективной и сюрьективной одновременно. 
\end{itemize}
