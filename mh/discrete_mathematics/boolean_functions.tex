\chapter{Булевы функции}

\section{Методы минимализации}
\subsection{Импликанты}
Литерал - это переменная или её отрицание. Н-р: $x_1, \overline{x_1}x_2$\\
Импликант $K$ - это такая коньюкция литералов функции $F$, что $K_i \rightarrow F_i$\\
Простой ипликант - это такой импликант, что вычеркиванием из него литералов нельзя получить новый импликант.\\
Н-р:\\
\begin{tabular}{ccccccc}
	$x_1$ & $x_2$ & $x_3$ & $K_1 = x_1$ & $K_2 = \overline{x_3}$ & $x_1x_2$ & $F$\\
	$0$ & $0$ & $0$ & $0$ & $1$ & $0$ & $0$\\
	$0$ & $0$ & $1$ & $0$ & $0$ & $0$ & $0$\\
	$0$ & $1$ & $0$ & $0$ & $1$ & $0$ & $1$\\
	$0$ & $1$ & $1$ & $0$ & $0$ & $0$ & $0$\\
	$1$ & $0$ & $0$ & $1$ & $1$ & $0$ & $1$\\
	$1$ & $0$ & $1$ & $1$ & $0$ & $0$ & $1$\\
	$1$ & $1$ & $0$ & $1$ & $1$ & $1$ & $1$\\
	$1$ & $1$ & $1$ & $1$ & $0$ & $1$ & $1$\\
\end{tabular}\\
$K_1$ - простой импликант\\
$K_2$ - не импликант\\
$K_3$ - импликант\\


\subsection{Сокращенные ДНФ}
\subsection{Тупиковые ДНФ}
\subsection{Кратчайшие и минимальные ДНФ}

\section{Классы булевых функций и полнота}
\subsection{Классы БФ}
\subsection{Теорема о функциональной полноте} 
