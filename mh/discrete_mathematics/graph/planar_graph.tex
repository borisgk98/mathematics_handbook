\section{Планарные графы}
\textbf{Плоский граф} - граф, который "нарисован" на плоскости так, чтобы ребра не пересекались.\\
\textbf{Планарный граф} - граф, который изоморфный плоскому.\\
\textbf{Грань плоского графа} - часть плоскости, границей которого являются его рёбра, и не содержащая внутри себя простых циклов.\\
\begin{center}\textbf{Формула Эйлера для плоских графов.}\end{center}
Если граф плоский, то выполняется такой равенство
$$
\mathbf{n - m + f = 2}
$$
где $n$ - число вершин, $m$ - число рёбер, $f$ - число граней.\\
Для любых планарных выполняется
$$
\mathbf{m \leq 3n - 6}
$$
где $n$ - число вершин, $m$ - число рёбер.\\
\textbf{Разбиением} графа $G$ называется граф, получающийся добавлением новой вершины на какое-нибудь ребро графа $G$.\\
Два графа называются гомеоморфными если получаются разбиением из одного и того же графа. (Стягиваем вершины степени 2 в ребро (удаляем их))\\
\begin{center}Критерий планарности. \textbf{Теорема Понтрягина-Куратовского}\end{center}
\textbf{Граф планарный тогда и только тогда, когда он не содержит подграфов, гомеоморфных $K_5$, $K_{3,3}$.}