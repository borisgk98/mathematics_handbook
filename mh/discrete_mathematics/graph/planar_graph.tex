\section{Планарные графы}
\textbf{Плоским} называется граф~$G = (V, E)$ такой, что:
\begin{itemize}
	\item $V \subset \mathbb R^2$;
	\item рёбра~--- (Жордановы) кривые, концами которых являются вершины;
	\item различные рёбра не имеют общих точек, за исключением концов.
\end{itemize}

Простыми словами \textbf{плоский граф} - граф, который "нарисован" на плоскости так, чтобы ребра не пересекались.

\textbf{Планарный граф} - граф, который изоморфный плоскому.

\begin{figure}[h!]
	\begin{center}
		\begin{tikzpicture}
		\GraphInit[vstyle=Classic]
		\SetGraphUnit{3}
		\Vertices{circle}{A,B,C,D,E,F}
		\Edges(A,B,C,D,A,C)
		\Edges(A,F,E,A)
		\SetVertexNoLabel
		\end{tikzpicture}
		
		\begin{tikzpicture}
		\GraphInit[vstyle=Classic]
		\SetGraphUnit{3}
		\Vertices{circle}{A,B,C,D,E,F}
		\Edges(A,B,C,D,A,C)
		\Edges(C,F,E,C)
		\SetVertexNoLabel
		\end{tikzpicture}
	\end{center}
	\caption{Плоский граф $G$ и изоморфный ему планарный граф}
	\label{pgraph}
\end{figure}

\textbf{Разбиением} графа $G$ называется граф, получающийся добавлением новой вершины на какое-нибудь ребро графа $G$.

Если $G$~--- граф и $G'$~--- плоский граф, изоморфный $G$, то $G'$ называется укладкой~$G$ в~$\mathbb R^2$.

Аналогично можно определить укладку графа в~$\mathbb R^3$, на сферу и т.\,д.

\begin{theorem}
	Любой граф можно уложить в~$\mathbb R^3$.
\end{theorem}
\begin{proof}
	Пусть $G = (V, E)$~--- граф, $V = \{ (1; 0; 0), (2; 0; 0), \ldots, (n; 0; 0) \}$.
	Рассмотрим плоскости, проходящие через~$Ox$ и~образующие с~плоскостью~$Oxy$ углы $\frac\pi2, \frac\pi{2\cdot2}, \ldots, \frac\pi{2m}$, где $m = |E|$.
	Получим плоский граф, т.\,к. плоскости пересекаются только по~прямой~$Ox$.
\end{proof}

\begin{theorem}
Граф укладывается на плоскость ровно тогда, когда он укладывается на сферу.
\end{theorem}
\begin{proof}
Пусть плоскость~$z = 0$ касается сферы в~точке~$O(0; 0; 0)$, $N$~--- точка на~сфере, диаметрально противоположная точке~$O$.
Для каждой точки сферы, не~совпадающей с~$N$, проведём прямую через~неё и~точку~$N$, которая пересечёт сферу и~плоскость, причём любые две из~этих прямых имеют единственную общую точку~$N$.
Получим биекцию между~точками сферы и~точками плоскости, тогда можно построить биекцию между укладками на~сфере и~укладками на~плоскости.
\end{proof}

Множество на~плоскости называется \textbf{линейно связным}, если любые две точки этого множества можно соединить кривой, целиком лежащей в~этом множестве.

\textbf{Гранью плоского графа}~$G = (V, E)$ называется часть множества~$\mathbb R^2 \setminus G$, которая линейно связна и не является подмножеством другого линейно связного множества.

\textbf{Грань плоского графа} - часть плоскости, границей которого являются его рёбра, и не содержащая внутри себя простых циклов. На рисунке \ref{pgraph} у графа $G$ есть 4 грани: между вершинами $ACD$, $ABC$, $AFE$ и та часть плоскости, которая окружает весь граф.

\begin{theorem}[формула Эйлера]
	В плоском связном графе $n - m + f = 2$, где $n, m, f$~--- число вершин, рёбер и граней соответственно.
\end{theorem}
\begin{proof}
	Рассмотрим остов данного графа.
	В~нём $n$~вершин, $n - 1$~рёбер и~1~грань.
	Формула Эйлера верна для~него: $n - (n - 1) + 1 = 2$.
	
	Добавим 1~ребро данного графа, тогда оно разобьёт одну грань на~две, т.\,е. число граней увеличится на~1.
	Формула Эйлера верна для~полученного графа.
	Повторяя $(m - (n - 1))$~раз, получим исходный граф, для~которого формула Эйлера верна.
\end{proof}

\begin{theorem}
	Пусть $G$~--- планарный граф с~$n \geqslant 3$~вершинами и~$m$~рёбрами. Тогда $m \leqslant 3n - 6$.
\end{theorem}
\begin{proof}
	При $m = 2$ неравенство выполняется.
	
	Пусть в графе $f$~граней, $m_i$~--- число рёбер в границе $i$\nobreakdash-й грани.
	Тогда $m_i \geqslant 3$, $\displaystyle \sum_{i=1}^f m_i \geqslant 3f$.
	С другой стороны, $\displaystyle \sum_{i=1}^f m_i = 2m$.
	По формуле Эйлера $n - m + f = 2 \Leftrightarrow f = m + 2 - n$.
	Получим:
	\begin{equation*}
	2m \geqslant 3f \Leftrightarrow 2m \geqslant 3m + 6 - 3n \Leftrightarrow m \leqslant 3n - 6
	\end{equation*}
\end{proof}

\begin{consequent}
	Планарный граф содержит хотя бы одну вершину со степенью, не большей 5.
\end{consequent}
\begin{proofcontra}
	Пусть $\forall v \in V \ deg v \geqslant 6$.
	Тогда 
\end{proofcontra}

\begin{theorem}
	Графы $K_5$ и $K_{3,3}$ не планарные.
\end{theorem}
\begin{proof}
	\begin{itemize}
		\item Рассмотрим $K_5$: $n = 5$, $m = 10$.
		Тогда $m \leqslant 3n - 6 \Leftrightarrow 10 \leqslant 9$.
		Неверно, значит, $K_5$ не планарен.
		\item Рассмотрим $K_{3,3}$.
		Пусть он планарный.
		В нём самый короткий цикл имеет длину~4.
		Тогда $2m \geqslant 4f \Leftrightarrow 2m \geqslant 4m + 8 - 4n \Leftrightarrow m \leqslant 2n - 4$.
		$n = 6$, $m = 9$, тогда $9 \leqslant 8$.
		Неверно, значит, $K_{3,3}$ не планарен.
	\end{itemize}
\end{proof}

Граф~$G' = (V', E')$ получается \textbf{подразбиением} ребра~$e = \{ u, v \}$ графа~$G = (V, E)$, если:
\begin{itemize}
	\item $V' = V \cup \{ u' \}$;
	\item $E' = (E \setminus \{ e \}) \cup \{ \{ u, u' \}, \{ v, u' \} \}$.
\end{itemize}

Графы $G$ и~$G'$ \textbf{гомеоморфны}, если они изоморфны графам, получающимся подразбиениями рёбер одного и~того~же графа (Стягиваем вершины степени 2 в ребро (удаляем их)).

\begin{figure}[h!]
	\begin{center}
		\begin{tikzpicture}
		\GraphInit[vstyle=Welsh]
		\SetGraphUnit{2}
		\Vertices{circle}{A,B,C,D}
		\Edges(A,B,C,D,A,C)
		\Edges(B,D)
		\SetVertexNoLabel
		\end{tikzpicture}
		\begin{tikzpicture}
		\GraphInit[vstyle=Welsh]
		\SetGraphUnit{2}
		\Vertices{circle}{A,B,C,D,F}
		\NO(A){E}
		\Edges(A,E,B,C,D,F,A,C)
		\Edges(B,D)
		\SetVertexNoLabel
		\end{tikzpicture}
	\end{center}
	\caption{Граф $G$ и получающийся из него двумя подразбиениями ($AB \Rightarrow AEB, \quad AD \Rightarrow AFD$) граф $G'$.}
	\label{pgraph}
\end{figure}

\begin{theorem}[Понтрягина-Куратовского]
	Граф~$G$ планарен ровно тогда, когда он не содержит подграфов, гомеоморфных $K_5$ или~$K_{3,3}$.
\end{theorem}
\begin{proof}
	Очевидно, что подграф планарного графа планарен.
	Если $G$~--- планарный граф, содержащий подграф $G'$, гомеоморфный $K_5$ или~$K_{3,3}$, то $G'$ тоже планарный, значит, $K_5$ или~$K_{3,3}$ планарен.
	Противоречие.
\end{proof}