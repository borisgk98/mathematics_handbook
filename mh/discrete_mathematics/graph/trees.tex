\section{Деревья}

Граф без~циклов называется \textbf{лесом}.

Связный лес называется \textbf{деревом}.

Ребро называется \textbf{мостом}, если при~его удалении увеличивается число компонент связности.

Дерево с $n$ вершинами, которым сопоставлены числа $1, \ldots, n$, называется \textbf{помеченным}.

\begin{statement}
Ребро~--- мост ровно тогда, когда оно не~содержится в~цикле.A
\end{statement}
\begin{proof}
\begin{enumerate}
	\item Докажем методом от противного, что если ребро содержится в цикле, то оно не является мостом. Пусть ребро $e$ содержится в цикле $W = v_0 e_0 \ldots u e v \ldots v_k$, $u'$ и $v'$~--- смежные вершины.
	\begin{enumerate}
		\item Если в этом маршруте нет ребра $e$, то при его удалении из графа $u'$ и $v'$ останутся смежными.
		\item Если $u' = v_0' e_0' \ldots u e v \ldots e_m v_m' = v'$~--- маршрут, соединяющий $u'$ и $v'$, тогда при удалении $e$ из графа $u'$ и $v'$ соединяет маршрут $u' = v_0' e_0' \ldots u \ldots e_0 v_0 = v_k e_{k-1} \ldots v \ldots e_m v_m' = v'$.
	\end{enumerate}
	\item Пусть $e = (u, v)$ не является мостом, тогда $u$, $v$ лежат в одной компоненте связности. Удалим $e$ из графа, тогда число компонент связности не изменилось, значит, $u$ и $v$ также лежат в одной компоненте связности, т./,е. существует цепь, соединяющая $u$ и $v$: $u = v_0 e_0 \ldots e_{k-1} v_k = v$. Тогда в исходном графе существует цикл $u = v_0 e_0 \ldots e_{k-1} v_k = v e u$.
\end{enumerate}
\end{proof}

\begin{theorem}
Следующие утверждения о графе $G$ с $n$ вершинами эквивалентны:
\begin{enumerate}
	\item $G$~--- дерево.
	\item $G$ связный и имеет $n - 1$ ребро.
	\item $G$ связный и каждое его ребро~--- мост.
	\item $G$ не содержит циклов и имеет $n - 1$ ребро.
	\item Любые две вершины графа $G$ соединены ровно одной простой цепью.
	\item $G$ не содержит циклов и добавление ребра приводит к появлению цикла.
\end{enumerate}
\begin{proof}
	\begin{itemize}
		\item Докажем 1) $\Rightarrow$ 3). Связность следует из определения дерева. В силу пред. утв. каждое ребро~--- мост.
		
		\item Докажем 3) $\Rightarrow$ 2). Связность по предположению. Докажем методом математической индукции, что в графе $n - 1$ ребро.
		\indbase Для $n = 1, 2$ очевидно.
		\indstep Пусть для графов с числом вершин, меньшим $n$,  Возьмём мост $e$ и удалим его. Получим две компоненты связности $G_1 = (V_1, E_1)$, $G_2 = (V_2, E_2)$. По предположению индукции $|E_1| = |V_1| - 1$, $|E_2| = |V_2| - 1$. В исходном графе рёбер $|E_1| + |E_2| + 1 = |V_1| + |V_2| - 1 = n - 1$.
		\indend
		
		\item Докажем 2) $\Rightarrow$ 4). В $G$ $n - 1$ ребро по предположению. Докажем методом математической индукции, что $G$ не содержит циклов.
		\indbase Для $n = 1, 2$ очевидно.
		\indstep Докажем, что в графе есть вершина степени 1. $\forall u \ deg u \geqslant 1$. $\forall u \ deg u \geqslant 2 \Rightarrow 2|E| = \sum_{u \in V} deg u \geqslant 2n \Rightarrow n - 1 = |E| \geqslant n$. Значит, в графе найдётся вершина степени 1. Удалим её и инцидентное ей ребро. Полученный граф содержит $n - 1$ вершину и удовлетворяет утверждению 2). По предположению индукции он не содержит циклов, тогда и исходный граф не содержит циклов.
		\indend
		
		\item Докажем 4) $\Rightarrow$ 5). Докажем связность методом математической индукции.
		\indbase Для $n = 1, 2$ очевидно.
		\indstep Пусть в графе $k$ компонент связности: $G_1 = (V_1, E_1)$, $G_2 = (V_2, E_2)$, \ldots, $G_k = (V_k, E_k)$. Они являются деревьями.
		\indend
		
		$|E_1| = |V_1| - 1$, $|E_2| = |V_2| - 1$, \ldots, $|E_k| = |V_k| - 1$. $n - 1 = |E_1| + \ldots + |E_k| = n - k \Rightarrow k = 1$, значит, граф связный.
		
		Пусть существуют вершины $u$, $v$ такие, что их соединяют две простые цепи, тогда в графе есть цикл, что противоречит предположению. Тогда эти вершины соединены ровно одной простой цепью.
		
		\item Докажем 5) $\Rightarrow$ 6). Предположим, что в графе есть цикл $v_0 e_0 v_1 e_1 \ldots v_k = v_0$, тогда есть две простые цепи $v_0 e_0 \ldots v_{k-1}$ и $v_{k-1} e_k v_k = v_0$, соединяющие $v_0$ и $v_{k-1}$, что противоречит предположению.
		
		Докажем, что добавление ребра приводит к появлению ровно одного цикла. Рассмотрим несоседних вершины $u$ и $v$. По предположению есть цепь $u = v_0 e_0 \ldots v_k = v$, соединяющая их. Тогда $u = v_0 e_0 \ldots v_k = v e u$~--- цикл, где $e$~--- $(u, v)$-маршрут. Пусть есть 2 цикла, соединяющих $u$ и $v$. Удалим $e$, цикл останется. Получили исходный граф, в котором нет циклов. Противоречие.
		
		\item 6) $\Rightarrow$ 1). Докажем связность. Рассмотрим вершины $u$ и $v$. Если они не соединены ребром, то соединим и по предположению получим цикл $v_0 e_0 \ldots u e v \ldots e_{k-1} v_k = v_0$. Тогда $u \ldots e_0 v_0 = v_k e_{k-1} \ldots v$~--- $(u, v)$-маршрут. Противоречие.
	\end{itemize}
\end{proof}
\end{theorem}

\subsection{Остовные деревья и методы нахождения минимальных остовных деревьев}


Остовом графа $G = (V, E)$ называется его подграф $G' = (V', E')$ такой, что $V = V'$ и $G'$~--- дерево.


\begin{statement}
	Любой связный граф содержит остов.
\end{statement}

\begin{statement}
	Если граф не является деревом, то в нём несколько остовов.
\end{statement}


Пусть $G = (V, E)$~--- граф. Весом называется функция $\alpha \colon E \to \mathbb{R}^+$. Весом графа называется $\sum_{e \in E} \alpha(e)$.

\textbf{Остовное дерево} графа состоит из минимального подмножества рёбер графа, таких, что из любой вершины графа можно попасть в любую другую вершину, двигаясь по этим рёбрам.\\
\textbf{Минимальное остовное дерево }(или минимальное покрывающее дерево) в связанном взвешенном неориентированном графе — это остовное дерево этого графа, имеющее минимальный возможный вес, где под весом дерева понимается сумма весов входящих в него рёбер.

\subsubsection{Алгоритмы поиска минимального остовного дерева}
\begin{itemize}
	\item \textbf{Алгоритм Крускала}\\
	Алгоритм Крускала изначально помещает каждую вершину в своё дерево, а затем постепенно объединяет эти деревья, объединяя на каждой итерации два некоторых дерева некоторым ребром. Перед началом выполнения алгоритма, все рёбра сортируются по весу (в порядке неубывания). Затем начинается процесс объединения: перебираются все рёбра от первого до последнего (в порядке сортировки), и если у текущего ребра его концы принадлежат разным поддеревьям, то эти поддеревья объединяются, а ребро добавляется к ответу. По окончании перебора всех рёбер все вершины окажутся принадлежащими одному поддереву, и ответ найден.\\
	\item \textbf{Алгоритм Прима}\\
	Искомый минимальный остов строится постепенно, добавлением в него рёбер по одному. Изначально остов полагается состоящим из единственной вершины (её можно выбрать произвольно). Затем выбирается ребро минимального веса, исходящее из этой вершины, и добавляется в минимальный остов. После этого остов содержит уже две вершины, и теперь ищется и добавляется ребро минимального веса, имеющее один конец в одной из двух выбранных вершин, а другой — наоборот, во всех остальных, кроме этих двух. И так далее, т.е. всякий раз ищется минимальное по весу ребро, один конец которого — уже взятая в остов вершина, а другой конец — ещё не взятая, и это ребро добавляется в остов (если таких рёбер несколько, можно взять любое). Этот процесс повторяется до тех пор, пока остов не станет содержать все вершины (или, что то же самое, n-1 ребро).
	
	В итоге будет построен остов, являющийся минимальным. Если граф был изначально не связен, то остов найден не будет (количество выбранных рёбер останется меньше n-1).
\end{itemize}


 


\subsection{Код Прюфера}

Каждому помеченному дереву можно взаимнооднозначно сопоставить последовательность из $(n - 2)$ чисел от $1$ до $n$, называемая \textbf{кодом Прюфера}. Алгоритм построения кода Прюфера для помеченного дерева $G = (V, E)$:
\begin{enumerate}
	\item Выбираем висячую вершину $v$ с наименьшим номером.
	\item Добавляем номер вершины, смежной с $v$, в код.
	\item Удаляем $v$ и ребро, инцидентное $v$, из дерева.
	\item Повторить, начиная с шага 1, $(n - 2)$ раза.
\end{enumerate}

\begin{statement}
	Различным помеченным деревьям соответствуют различные коды Прюфера.
\end{statement}
\begin{proofmathind}
	\indbase При $n = 3$ легко проверить.
	\indstep Пусть утверждение верно при $n \geqslant 3$, $G = (V, E)$ и $G' = (V', E')$~--- различные помеченные деревья с $(n + 1)$ вершинами в каждом.
	\begin{enumerate}
		\item Пусть в $G$ и $G'$ вершины с наименьшим номером смежны с вершинами с различными номерами.
		\item Пусть в $G$ и $G'$ вершины с наименьшим номером смежны с вершинами с одинаковыми номерами. Выполняем шаг построения кода, тогда оставшиеся деревья различны, значит, по предположению индукции у них различные коды.
	\end{enumerate}
	\indend
\end{proofmathind}

	Алгоритм построения дерева по коду $A_0 = (a_1, \ldots, a_{n-2})$.
\begin{enumerate}
	\item Пусть $B_0 = (1, \ldots, n)$.
	\item Находим наименьшее $b \in B_i \colon b \notin A_i$. Тогда в дереве есть ребро $\{ b, a_i \}$. $A_{i+1} = A_i \setminus \{ a_i \}$, $B_{i+1} = B_i \setminus \{ b \}$.
	\item Повторяем шаг 2 $(n - 2)$ раз. Получим $B_{n-2} = \{ b', b'' \}$, значит, в дереве есть ребро $\{ b', b'' \}$.
\end{enumerate}

\begin{statement}
	Указанный алгоритм построения дерева по коду из $n$ чисел строит дерево.
\end{statement}
\begin{proofmathind}
	\indbase При $n = 1$ легко проверить.
	\indstep Рассмотрим графы $T_1, \ldots, T_{n-1}$, полученные в процессе построения дерева. $T_1$ не содежрит циклов. $T_2$ получается из $T_1$ либо добавлением новой вершины, либо добавлением моста, что не приводит к появлению цикла.
	
	$T_{n-1}$ не содержит циклов и содержит $n$ вершин и $(n - 1)$ ребёр, значит, $T_{n-1}$~--- дерево.
	\indend
\end{proofmathind}

\begin{theorem}[Кэли]
	Пусть $G = (V, E)$~--- дерево, $n = |V|$, вершинам $G$ сопоставлены числа $1, \ldots, n$. Всего можно составить $n^{n-2}$ таких неизоморфных деревьев.
\end{theorem} % коды прюфера