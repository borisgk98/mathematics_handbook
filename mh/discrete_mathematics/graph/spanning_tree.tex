\section{Остовные деревья}
\textbf{Остовное дерево} графа состоит из минимального подмножества рёбер графа, таких, что из любой вершины графа можно попасть в любую другую вершину, двигаясь по этим рёбрам.\\
\textbf{Минимальное остовное дерево }(или минимальное покрывающее дерево) в связанном взвешенном неориентированном графе — это остовное дерево этого графа, имеющее минимальный возможный вес, где под весом дерева понимается сумма весов входящих в него рёбер.
\begin{center}
	\textbf{Алгоритмы поиска минимального остовного дерева}
\end{center}
\begin{itemize}
	\item \textbf{Алгоритм Крускала}\\
	Алгоритм Крускала изначально помещает каждую вершину в своё дерево, а затем постепенно объединяет эти деревья, объединяя на каждой итерации два некоторых дерева некоторым ребром. Перед началом выполнения алгоритма, все рёбра сортируются по весу (в порядке неубывания). Затем начинается процесс объединения: перебираются все рёбра от первого до последнего (в порядке сортировки), и если у текущего ребра его концы принадлежат разным поддеревьям, то эти поддеревья объединяются, а ребро добавляется к ответу. По окончании перебора всех рёбер все вершины окажутся принадлежащими одному поддереву, и ответ найден.\\
	\item \textbf{Алгоритм Прима}\\
	Искомый минимальный остов строится постепенно, добавлением в него рёбер по одному. Изначально остов полагается состоящим из единственной вершины (её можно выбрать произвольно). Затем выбирается ребро минимального веса, исходящее из этой вершины, и добавляется в минимальный остов. После этого остов содержит уже две вершины, и теперь ищется и добавляется ребро минимального веса, имеющее один конец в одной из двух выбранных вершин, а другой — наоборот, во всех остальных, кроме этих двух. И так далее, т.е. всякий раз ищется минимальное по весу ребро, один конец которого — уже взятая в остов вершина, а другой конец — ещё не взятая, и это ребро добавляется в остов (если таких рёбер несколько, можно взять любое). Этот процесс повторяется до тех пор, пока остов не станет содержать все вершины (или, что то же самое, n-1 ребро).
	
	В итоге будет построен остов, являющийся минимальным. Если граф был изначально не связен, то остов найден не будет (количество выбранных рёбер останется меньше n-1).
\end{itemize}