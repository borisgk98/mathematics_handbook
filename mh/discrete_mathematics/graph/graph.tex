\section{Основные понятия}
\textbf{Граф} — абстрактный математический объект, представляющий собой множество вершин графа и набор рёбер, то есть соединений между парами вершин. Например, за множество вершин можно взять множество аэропортов, обслуживаемых некоторой авиакомпанией, а за множество рёбер взять регулярные рейсы этой авиакомпании между городами. Пример графа:

\begin{tikzpicture}
\GraphInit[vstyle=Welsh]
\SetGraphUnit{2}
\Vertices{circle}{A,B,C,D}
\Edges(A,B,C,D,A,C)
\SetVertexNoLabel
\end{tikzpicture}

Формальное определение:

\textbf{Графом} называется пара множеств $G = (V, E)$, где $V$~--- множество вершин графа, $E \subseteq V^2$~--- множество рёбер графа.

Если $e = \{ u, v \}$, $e \in E$, то говорят, что:
\begin{itemize}
	\item ребро~$e$ соединяет вершины~$u$ и~$v$;
	\item $u$ и~$v$~--- концы ребра~$e$;
	\item ребро~$e$ инцидентно вершинам $u$ и~$v$;
	\item вершины $u$ и~$v$ \textbf{инцидентны} ребру~$e$.
\end{itemize}

На рисунках вершины графа изображают точками, а~рёбра $e = \{ u, v \}$~--- кривыми, соединяющими точки, которые изображают вершины $u$ и~$v$.

Вершины называются \textbf{соседними}, если их соединяет ребро, иначе~--- \textbf{несоседними}.

Ребро вида $e = \{ u, u \}$ называется \textbf{петлёй}.

Граф, в~котором любые две вершины соединены ребром, называется \textbf{полным} и~обозначается $K_n$, где $n$~--- число вершин в~нём.

Графы $G_1 = (V_1, E_1)$ и~$G_2 = (V_2, E_2)$ называются \textbf{изоморфными}, если существует биекция $\varphi \colon V_1 \to V_2$ такая, что
$\forall u, v \in V_1 \ \allowbreak ((u, v) \in E_1 \opbr\Leftrightarrow (\varphi(u), \varphi(v)) \in E_2)$, иначе~--- \textbf{неизоморфными}. Иными словами два графа называются \textbf{изоморфными}, если они одинаковые с точностью до переименования вершин.

$\varphi$ называется \textbf{изоморфизмом}.

Число рёбер в~графе~$G$, инцидентных вершине~$u$, называется \textbf{степенью} вершины и~обозначается $\deg_G u$.


\begin{lemma}[о рукопожатиях]
	\[ \sum_{u \in V} \deg_G u = 2|E| \]
	где $G = (V, E)$~--- граф.
\end{lemma}
\begin{proofmathind}
	\indbase $|E| = 0$: в~таком графе $\displaystyle \sum_{u \in V} \deg u = 0$.
	\indstep Пусть лемма верна для~$|E| = n$.
	Докажем её для~$|E| = n + 1$.
	Для~этого достаточно заметить, что каждое новое ребро увеличивает степени двух вершин на~1.
	\indend
\end{proofmathind}


\textbf{Маршрутом} в~графе~$G = (V, E)$ называется последовательность вершин и~рёбер вида $v_1 e_1 v_2 \ldots e_k v_{k+1}$, где $e_i = \{ v_i, v_{i+1} \}$.

Маршрут, в~котором все рёбра различны, называется \textbf{цепью}.

Цепь, в~которой все вершины, за~исключением, может быть, первой и~последней, различны, называется \textbf{простой}.

Маршрут, в~котором первая и~последняя вершины совпадают, называется \textbf{замкнутым}.

Замкнутая цепь называется \textbf{циклом}.

Маршрут, соединяющий вершины $u$ и~$v$, называется \textbf{$(u, v)$\nobreakdash-\hspace{0pt}маршрутом}.


\begin{lemma}
	$(u, v)$\nobreakdash-\hspace{0pt}маршрут содержит $(u, v)$\nobreakdash-\hspace{0pt}простую цепь.
\end{lemma}
\begin{proof}
	Пусть $u = v_1 e_1 v_2 \ldots e_k v_{k+1} = v$~--- не~простая цепь, тогда $\exists i < j \colon v_i = v_j$.
	Уберём из~маршрута подпоследовательность $e_i v_{i+1} \ldots e_{j-1} v_j$, получим маршрут, в~котором совпадающих вершин на~одну меньше.
	Повторяя, получим простую цепь, являющуюся частью данного маршрута.
\end{proof}

\begin{lemma}
	Любой цикл содержит простой цикл.
\end{lemma}%
Доказательство аналогично предыдущему.

\begin{lemma}
	Если в~графе есть две различные простые цепи, соединяющие одни и~те~же вершины, то в~этом графе есть простой цикл.
\end{lemma}
\begin{proof}
	Пусть $u = v_1 e_1 v_2 \ldots e_n v_{n+1} = v$, $u = v_1' e_1' v_2' \ldots e_m' v_{m+1}' = v$~--- простые цепи.
	Найдём наименьшее~$i \colon e_i \neq e_i'$, тогда $v_i e_i v_{i+1} \ldots e_n v_{n+1} = v_{m+1}' e_m' \ldots e_i' v_i' = v_i$~--- цикл, значит, можно получить простой цикл.
\end{proof}



