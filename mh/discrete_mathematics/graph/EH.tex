\section{Связность графов}

Вершины $u$ и~$v$ называются \textbf{связанными}, если существует $(u, v)$-маршрут, иначе~---
\textbf{несвязанными}.

Граф называется \textbf{связным}, если в~нём любые две вершины связаны, иначе~--- \textbf{несвязным}.

Граф~$G' = (V', E')$ называется \textbf{подграфом} графа~$G = (V, E)$, если $V' \subseteq V$ и
~$E' \subseteq E$.

\textbf{Компонентой связности} графа называется его максимальный (относительно включения)
связный подграф.

\begin{figure}[h!]
	\begin{center}
		\begin{tikzpicture}
		\GraphInit[vstyle=Welsh]
		\SetGraphUnit{2}
		\Vertices{circle}{A,B,C,D,E,F,G}
		\Edges(A,B,C,D,A,C,D,B)
		\Edges(E,F,G,E)
		\SetVertexNoLabel
		\end{tikzpicture}
	\end{center}
	\caption{Граф с двумя компонентами связанности}
\end{figure}

\input{./mh/discrete_mathematics/graph/EH/eulerian_graphs} 

\input{./mh/discrete_mathematics/graph/EH/hamiltonian_graphs}
