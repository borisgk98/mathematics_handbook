\section{Связность графов}

Вершины $u$ и~$v$ называются \textbf{связанными}, если существует $(u, v)$-маршрут, иначе~---
\textbf{несвязанными}.

Граф называется \textbf{связным}, если в~нём любые две вершины связаны, иначе~--- \textbf{несвязным}.

Граф~$G' = (V', E')$ называется \textbf{подграфом} графа~$G = (V, E)$, если $V' \subseteq V$ и
~$E' \subseteq E$.

\textbf{Компонентой связности} графа называется его максимальный (относительно включения)
связный подграф.

\begin{figure}[h!]
	\begin{center}
		\begin{tikzpicture}
		\GraphInit[vstyle=Welsh]
		\SetGraphUnit{2}
		\Vertices{circle}{A,B,C,D,E,F,G}
		\Edges(A,B,C,D,A,C,D,B)
		\Edges(E,F,G,E)
		\SetVertexNoLabel
		\end{tikzpicture}
	\end{center}
	\caption{Граф с двумя компонентами связанности}
\end{figure}

\subsection{Эйлеровы графы}
Цикл, содержащий все рёбра графа, называется \textbf{эйлеровым}.

Граф, содержащий эйлеров цикл, называется \textbf{эйлеровым}.



\begin{figure}[h!]
	\begin{center}
		\begin{tikzpicture}
		\GraphInit[vstyle=Welsh]
		\SetGraphUnit{1.5}
		\Vertices{circle}{A,B,C,D,E,F}
		\Edges(A,B,C,D,E,F,A,C,E,A)
		\SetVertexNoLabel
		\end{tikzpicture}
		\begin{tikzpicture}
		\GraphInit[vstyle=Welsh]
		\SetGraphUnit{1.5}
		\Vertices{circle}{A,B,C,D,E,F}
		\Edges(A,B,C,D,E,F,A,C,E,A)
		\Edges(B,D)
		\SetVertexNoLabel
		\end{tikzpicture}
		\begin{tikzpicture}
		\GraphInit[vstyle=Welsh]
		\SetGraphUnit{1.5}
		\Vertices{circle}{A,B,C,D,E,F}
		\Edges(A,B,C,D,E,F,A,C,E,A)
		\Edges(B,D)
		\Edges(C,F)
		\SetVertexNoLabel
		\end{tikzpicture}
	\end{center}
	\caption{Эйлеров граф, полуэйлеров граф и граф не являющийся ни эйлеровым ни полуэйлеровым}
\end{figure}

\begin{theorem}
Связный граф эйлеров $\Leftrightarrow$ степени всех вершин чётны.
\end{theorem}
\begin{proof}
\begin{enumerate}
	\item $\Rightarrow$. Пусть в~графе есть эйлеров цикл.
	Выберем вершину~$v_0$ в~этом цикле и~начнём обходить его.
	При~каждом посещении вершины~$v \neq v_0$ её степень увеличивается на~2.
	То есть, если посетить её $k$ раз, то $\deg v = 2k \div 2$.
	
	Для~$v_0$ степень увеличивается на~1 в~начале обхода, на~1 в~конце обхода и~на~2 при~промежуточных посещениях.
	Т.\,о., её степень чётна.
	
	\item $\Leftarrow$. Пусть степени всех вершин чётны.
	Выберём цепь~$C = (v_0; e_0; v_1; e_1; \ldots; e_{k-1}; v_k)$ наибольшей длины.
	Все рёбра, инцидентные~$v_k$, присутствуют в~этой цепи, иначе её можно было~бы удлинить.
	
	Докажем методом от~противного, что $v_0 = v_k$.
	Пусть $v_0 \neq v_k$.
	При прохождении вершины~$v_i = v_k$, где $0 < i < k$, степень~$v_k$ увеличивается на~2.
	Также проходим по~ребру~$e_{k-1}$, тогда степень~$v_k$ нечётна.
	Противоречие.
	
	Докажем методом от~противного, что $C$ содержит все рёбра.	
	Пусть найдётся ребро~$e = \{ u, v \}$, не~входящее в~$C$.
	Возьмём первое ребро~$e' = \{ v_i, v' \}$ из~$(v_0, u)$\nobreakdash-\hspace{0pt}маршрута, не~входящее в~$C$.
	Тогда цепь~$(v'; e'; v_i; e_i; \ldots; e_{k-1}; v_k = v_0; e_0; v_1; e_1; \ldots; v_{i-1})$ длиннее, чем~$C$.
	Противоречие.
\end{enumerate}
\end{proof}

\subsubsection{Алгоритмы нахождения эйлерова цикла}
\begin{enumerate}
	\item\textbf{Алгоритм Флёри (очень медленный)}.
	\begin{enumerate}
		\item Выберем произвольную вершину.
		\item Пусть находимся в~вершине~$v$.
		Выберем ребро, инцидентное ей, которое должно быть мостом, только если не~осталось других рёбер.
		\item Проходим по~выбранному ребру и~вычёркиваем его.
		\item Повторяем, пока есть рёбра.
	\end{enumerate}
	\item\textbf{Алгоритм объединения циклов}.
	\begin{enumerate}
		\item\label{list:cycle_alg_step1} Выберем произвольную вершину.
		\item Выбираем любое непосещённое ребро и~идём по~нему.
		\item\label{list:cycle_alg_step3} Повторяем, пока не~вернёмся в~начальную вершину.
		\item\label{list:cycle_alg_step4} Получим цикл~$C$.
		Если он не~эйлеров, то $\exists u \in C, \ e = \{ u, u' \} \colon u' \notin C$.
		Повторяем шаги \ref{list:cycle_alg_step1}--\ref{list:cycle_alg_step3} для~начальной вершины~$u$.
		Получим цикл~$C'$, рёбра которого не~совпадают с~рёбрами $C$.
		Объединим эти циклы и~получим новый.
		Повторяем шаг \ref{list:cycle_alg_step4}.
	\end{enumerate}
\end{enumerate}

Цепь называется \textbf{эйлеровым путём}, если она не~является циклом и~содержит все рёбра.

Граф называется \textbf{полуэйлеровым}, если в~нём есть эйлеров путь.

\begin{theorem}
Связный граф полуэйлеров $\Leftrightarrow$ степени двух вершин нечётны, а остальных~--- чётны.
\end{theorem}
\begin{proof}
\begin{enumerate}
	\item $\Rightarrow$. Пусть в~графе есть эйлеров путь.
	Соединив его концы ребром, получим эйлеров цикл.
	Степени соединённых вершин увеличились каждая на~1, значит, они были нечётными, а степени остальных вершин~--- чётными.
	\item $\Leftarrow$. Пусть степени двух вершин нечётны, а остальных~--- чётны.
	Соединим нечётные вершины ребром, тогда можно получить эйлеров цикл.
	Убрав из~него добавленное ребро, получим эйлеров путь.
\end{enumerate}
\end{proof} 

% beta
\subsection{Гамильтоновы графы}
Простой цикл, содержащий все вершины графа, называется \textbf{гамильтоновым}.

Граф называется \textbf{гамильтоновым}, если в~нём есть гамильтонов цикл.

\begin{theorem}[Дирака]
\label{th:Dirac}
Если в~графе~$G = (V, E)$ с~$n \geqslant 3$~вершинами $\forall u \in V \ \deg u \geqslant \frac{n}2$, то граф гамильтонов.
\end{theorem}
\begin{proof}
\begin{enumerate}
	\item Докажем методом от~противного, что граф связный.
	Пусть он несвязный.
	Выберем компоненту связности~$G' = (V', E')$ с~наименьшим числом вершин, тогда $|V'| \leqslant \frac{n}2$.
	Возьмём $v \in V'$, тогда $\deg v \leqslant |V'| - 1 < \frac{n}2$.
	Противоречие с~условием.
	\item Выберем цепь~$C = (v_0; e_0; v_1; \ldots; e_{k-1}; v_k)$ максимальной длины.
	Тогда все вершины, соседние с~$v_0$, лежат в~этой цепи, иначе можно увеличить длину цепи.
	Среди $v_1, v_2, \ldots, v_k$ не~менее $\frac{n}2$~вершин, соседних с~$v_0$, т.\,к. $\deg v_0 \geqslant \frac{n}2$.
	Аналогично для~$v_k$.
	
	Найдутся $v_{i-1}$ и~$v_i$ такие, что $v_{i-1}$ соседняя с~$v_k$, а $v_i$~--- с~$v_0$.
	
	Докажем, что $(v_i; e_{i+1}; \ldots; v_k; e; v_{i-1}; e_{i-1}; \ldots; v_0; e'; v_i)$~--- гамильтонов цикл, методом от~противного.
	Предположим обратное, тогда есть вершина~$u$, не~входящая в~цикл, и~существует $(v_0, u)$\nobreakdash-\hspace{0pt}маршрут.
	Значит, существует ребро, инцидентное одной из~вершин цикла, но не~входящее в~него, и~можно получить более длинную цепь.
	Противоречие.
\end{enumerate}
\end{proof}

\begin{theorem}[Оре]
	Если в~графе с~$n \geqslant 3$~вершинами для~любых двух несмежных вершин $u$ и~$v$ $deg u + deg v \geqslant n$, то граф гамильтонов.
\end{theorem}
\begin{proof}
\begin{enumerate}
	\item Докажем методом от~противного, что граф связный.
	Пусть он несвязный, тогда в~нём найдутся хотя~бы две компоненты связности $G_1(V_1, E_1)$ и~$G_2(V_2, E_2)$.
	Пусть $u \in V_1$, $v \in V_2$. $u$ и~$v$ несмежные, тогда
	\begin{equation*}
	\deg u \leqslant |V_1| - 1, \ \deg v \leqslant |V_2| - 1 \opbr\Rightarrow \deg u + \deg v opbr\leqslant |V_1| + |V_2| - 2 \opbr\leqslant n - 2
	\end{equation*}
	
	Противоречие с~условием.
	
	\item Докажем, что граф гамильтонов.
	Выберем цепь~$W = (v_0; e_0; v_1; \ldots; e_{k-1}; v_k)$ наибольшей длины.
	В~ней содержатся все вершины, соседние с~$v_0$ или~с~$v_k$.
	Т.\,о., среди вершин $v_1, \ldots, v_k$ $deg v_0$ соседних с~$v_0$.
	Аналогично для~$v_k$.
	
	$\deg v_0 + \deg v_k \geqslant n$, тогда найдутся $v_i$ и~$v_{i+1}$ такие, что $v_i$ соседняя с~$v_k$, а $v_{i+1}$~--- с~$v_0$.\newline
	$(v_{i+1}; e_{i+1}; \ldots; v_k; e; v_i; e_{i-1}; v_{i-1}; \ldots; e_0; v_0; e'; v_{i+1})$~--- гамильтонов цикл (доказательство аналогично доказательству в теореме \ref{th:Dirac} (Дирака)).
\end{enumerate}
\end{proof}
