\section{Эйлеровы и Гамильтоновы пути и циклы}


\textbf{Эйлеров путь} — это путь, проходящий по всем рёбрам графа и притом только по одному разу. \\
\textbf{Эйлеров цикл} — эйлеров путь, являющийся циклом. То есть замкнутый путь, проходящий через каждое ребро графа ровно по одному разу.\\
\textbf{Эйлеров граф} — граф, содержащий эйлеров цикл.\\
\textbf{Полуэйлеров граф} — граф, содержащий эйлеров путь.\\
\begin{center}
	\textbf{Существование эйлерова цикла и эйлерова пути}
\end{center}
\begin{itemize}
	\item В неориентированном графе\\
	Согласно теореме, доказанной Эйлером, эйлеров цикл существует тогда и только тогда, когда граф связный и в нём отсутствуют вершины нечётной степени.
	
	Эйлеров путь в графе существует тогда и только тогда, когда граф связный и содержит не более двух вершин нечётной степени.[1][2] Ввиду леммы о рукопожатиях, число вершин с нечётной степенью должно быть четным. А значит эйлеров путь существует только тогда, когда это число равно нулю или двум. Причём когда оно равно нулю, эйлеров путь вырождается в эйлеров цикл.
\end{itemize}


\textbf{Гамильтоновым циклом} является такой цикл, который проходит через каждую вершину данного графа ровно по одному разу.\\
\textbf{Гамильтонов граф} - граф, содержащий гамильтонов цикл.\\
\begin{center}
	\textbf{Условия существования гамильтонова цикла в графе}
\end{center}
\begin{itemize}
	\item \textbf{Условие Дирака}\\
	Пусть $p$ — число вершин в данном графе и $p>3$. Если степень каждой вершины не меньше, чем $\frac{p}{2}$, то данный граф — гамильтонов. 
	
	\item \textbf{Условие Оре}\\	
	Пусть $p$ — количество вершин в данном графе и $p>2$. Если для любой пары несмежных вершин $(x, y)$ выполнено неравенство $\deg x + \deg y\geqslant p$, то данный граф — гамильтонов (другими словами: сумма степеней любых двух несмежных вершин не меньше общего числа вершин в графе).
\end{itemize}
