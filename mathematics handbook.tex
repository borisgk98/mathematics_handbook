\documentclass[a4paper,12pt]{report}
\usepackage[utf8]{inputenc}
\usepackage[russian]{babel}
\usepackage{graphicx}

\usepackage[left=2cm,top=2cm,right=2cm,bottom=2cm]{geometry}

\usepackage{amsmath} %для работы больших скобок в формулах и др. фич
\usepackage{amsfonts}
\usepackage{amssymb}

\usepackage{tikz} %для графиков ф-ий
\usepackage{pgfplots}
\usepackage{tkz-graph} %для графов
\usepackage{tkz-berge}

\usepackage{indentfirst} % отступ красной строки
\usepackage[unicode,colorlinks=false]{hyperref} % ссылки в оглавлении
\usepackage{mathenvrus}

\usepackage{graphicx} % для figure


\begin{document}

\author{Борис Кожуховский}
\title{Справочник по математике}
\date{2017}

\maketitle
\tableofcontents

%inlude - ставит на новую страницу
%input - просто продолжает (как будто вставляет текст из одного файла в другой)
%нельзя делать несколько вложенностей include, а input можно
\chapter*{Введение}
Цели этого справочника:\\
\begin{itemize}
	\item Систематизация и сохранение математических знаний, полученных мной за годы учёбы
	\item Сбор информации, которую трудно найти в понятном мне виде
	\item Конспектирование лекций в красивом виде
	\item Изучение LaTeX
\end{itemize} 
\chapter*{Благодарность}
Автор это справочника благодарит Максима Гунбина за предоставление некоторых материалов в уже написанном в TeX виде.

\part{Алгебра}   
\chapter{Рациональные выражения}
\section{Формулы сокращённого умножения}
\begin{itemize}
	\item $(a \pm b)^n$ вычисляется через треугольник паскаля\\
	\begin{center}
		\includegraphics[scale=0.2]{./mh/algebra/rational_expressions/Pascal_triangle.png}
	\end{center}
	Например:\\
	\begin{itemize}
		\item 
	\end{itemize}
\end{itemize}           

\part{Математический анализ}
\chapter{Функции и их свойства.}
\section{График функции}
\textbf{Преобразование графиков ф-ий:}

\begin{enumerate}
	
	\item \textbf{Симметрия относительно осей координат}
	\begin{itemize}
		\item 
		Функции $y = f(x)$ и $y = -f(x)$ имеют одну и ту же область определения, их графики симметричны относительно оси $Ox$.
		%\includegraphics[scale=1]{./mh/math_analysis/functions/graph1.png}\\
		\item 
		Функции $y = f(x)$ и $y = f(-x)$ имеют области определения, симметричные относительно точки $O$. Их графики симметричны относительно оси $Oy$.
		%\includegraphics[scale=1]{./mh/math_analysis/functions/graph2.png}\\
	\end{itemize}

	\item \textbf{Сдвиг вдоль осей координат (параллельный перенос)}
	\begin{itemize}
		\item 
		Функция $y = f(x - a)$, где $a \neq 0$, определена для всех $x$, таких, что $(x - a) \in D(f(x))$. График ф-ии $y = f(x - a)$ 
		получается сдвигом вдоль оси $Ox$ на величину $|a|$ графика функции $y = f(x)$ вправо, если $a > 0$, 
		и влево, если $a < 0$.
		\item
		Функция $y = f(x) + B$, где $B \neq 0$, имеет ту же область определения, что и ф-ия $y = f(x)$. График ф-ии $y = f(x) + B$ 
		получается сдвигом вдоль оси $Oy$ на величину $|B|$ графика функции $y = f(x)$ вверх, если $B > 0$, 
		и вниз, если $B < 0$.
	\end{itemize}
	
	\item \textbf{Растяжение с сжатие графика вдоль всей оси координат}\\
	
	\item \textbf{Построение графика функции $y = Af(k(x - a)) + B)$ по графику функции $y = f(x)$}\\
	
	\item \textbf{Симметрия относительно прямой $y = x$}\\
	
\end{enumerate}

 %график фнкции
\section{Периодичность ф-ий}
\textbf{Определение:}\\
\textbf{\fbox{\parbox{15cm}{Функцию $y = f(x)$ с областью определения $X$ называют периодической, если 
	$\exists T \neq 0 \quad \forall x \in X$ такой, что $(x + T) \in X$, и $(x - t) \in X$, и $f(x + T) = f(x)$}}}\\

\textbf{Пример ур-ия, где используется периодичность ф-ий:}\\
Пусть $f(x)$ - периодическая функция с периодом 8, такая, что $f(x) = 8x - x^{2}$ при $x \in [0; 8)$. Решите уравнение $f(2x + 16) + 23 = 5f(x)$.\\
Решение:\\
\begin{enumerate}
	\item$$\begin{cases}
				f(x) = f(x + T) = f(x - T)\\
				T = 8\\
			\end{cases}
		 \quad \Longrightarrow \quad f(2x + 16) = f(2x)$$
	\item $x \in [0; 4) \quad \Longrightarrow \quad 2x \in [0; 8)$\\
		Решаем уравнение для этого случая:\\
		$f(2x) + 23 = 5f(x)$\\
		$16x - 4x^{2} + 23 = 40x - 5x^{2}$\\
		$x^{2} - 24x + 23 = 0$\\
		$x1 = 1$\\
		$x2 = 23\qquad \mbox{побочный корень для }x \in [0; 4)$\\
	\item $x \in [4; 8) \quad \Longrightarrow \quad (2x - 8) \in [0; 8)$\\
	Решаем уравнение для этого случая:\\
	$f(2x - 8) + 23 = 5f(x)$\\
	$16x - 64 - 4x^{2} + 16x - 64  + 23 = 40x - 5x^{2}$\\
	$x^{2} - 8x - 105 = 0$\\
	$x1 = 7$\\
	$x2 = -15\qquad \mbox{побочный корень для }x \in [4; 8)$\\
			
	\item Так как наша функция имеет период 8, то и корни будут повторятся с такой же периодичностью, так как $f(x) = f(x + T) = f(x - T)$.
	 То есть получаем корни $x = 1 + 8n$ и $x = 7 + 8n$.
\end{enumerate}
Ответ: $x = 1 + 8n$ и $x = 7 + 8n$. %периодичность
\chapter{Пределы}
\input{./mh/math_analysis/limits/limit_at_infinity}
\chapter{Производная}
\textbf{Определение:}\\
\textbf{Производной функции в точке называется предел отношения приращения функции к приращению аргумента, когда приращение аргумента стремится к 0.}\\
\textbf{\fbox{\parbox{15cm}{
			$$
			\mathbf{
			f'(x_{0}) = \lim_{\triangle x \to 0}{\frac{\triangle f}{\triangle x}} = 
			\lim_{\triangle x \to 0}{\frac{f(x_0 + \triangle x) - f(x_0)}{\triangle x}}
	     	}
			$$
			}}}\\\\
$\triangle x \qquad$ - приращение аргумента, то есть изменение аргумента от $x$ до $x_0$ (дельта $x$).\\
$\triangle f = f(x + \triangle x) - f(x) \qquad$ - приращение функции (дельта $f$).\\

\section{Свойства производных}
\textbf{
	\begin{enumerate}
		\item $ \mathbf{(C * x)' = C * (x)' \qquad C = const } $
		\item $ \mathbf{(f + g)' = f' + g' }$
		\item $ \mathbf{(f * g)' = f' * g + g' * f }$
		\item $ \mathbf{ \left(\dfrac{f}{g}\right)' = \dfrac{f' * g - g' * f}{g^2} } $
		\item $ \mathbf{ (f(g))' = f'(g) * g'(f) } $
		\item $ \mathbf{ (f^g) = f^g * \ln{f} * g' + g * f^(g - 1) * f' } \qquad $
		\item $ \mathbf{ f'(y) = \dfrac{1}{g(x)} } \qquad $
		 $\mathbf{f(y)}$ и $\mathbf{g(x)}$ - взаимообратные функции ($\mathbf{D(f(y)) = E(g(x))}$ и $\mathbf{D(g(x)) = E(f(y))}$).
	\end{enumerate}
}

\input{./mh/math_analysis/derivative/functions_extremes/2_variables}
\input{./mh/math_analysis/derivative/functions_extremes/3_variables}
\input{./mh/math_analysis/derivative/functions_extremes/conditional_extremum.tex}



\section{Геометрическая интерпретация производной}
\input{./mh/math_analysis/geometric/tangent}
\input{./mh/math_analysis/geometric/normal}

\section{Геометрическая интерпретация производной}
\subsection{Касательная}
\subsubsection{В трёхмерном пространстве}
Пусть дана функция, задающая поверхность $F(x,y,z) = 0$.\\
\textbf{Касательная плоскость} к поверхности в точке $M_0$ – это плоскость, содержащая касательные ко всем кривым, которые принадлежат данной поверхности и проходят через точку $M_0$. Её уравнение имеет вид $F'_x(M_0)(x - x_0) + F'_y(M_0)(y - y_0) + F'_z(M_0)(z - z_0) = 0$.
\subsection{Нормаль}
\subsubsection{В трёхмерном пространстве}
Пусть дана функция, задающая поверхность $F(x,y,z) = 0$.\\
\textbf{Нормаль} к поверхности в точке $M_0$ – это прямая, проходящая через данную точку перпендикулярно касательной плоскости. Её каноническое уравнение имеет вид $\frac{x - x_0}{F'_x(M_0)} = \frac{y - y_0}{F'_y(M_0)} = \frac{z - z_0}{F'_z(M_0)}$.

\part{Дискретная математика}             
\chapter{Булевы функции}

\section{Методы минимализации}
\subsection{Импликанты}
Литерал - это переменная или её отрицание. Н-р: $x_1, \overline{x_1}x_2$\\
Импликант $K$ - это такая коньюкция литералов функции $F$, что $K_i \rightarrow F_i$\\
Простой ипликант - это такой импликант, что вычеркиванием из него литералов нельзя получить новый импликант.\\
Н-р:\\
\begin{tabular}{ccccccc}
	$x_1$ & $x_2$ & $x_3$ & $K_1 = x_1$ & $K_2 = \overline{x_3}$ & $x_1x_2$ & $F$\\
	$0$ & $0$ & $0$ & $0$ & $1$ & $0$ & $0$\\
	$0$ & $0$ & $1$ & $0$ & $0$ & $0$ & $0$\\
	$0$ & $1$ & $0$ & $0$ & $1$ & $0$ & $1$\\
	$0$ & $1$ & $1$ & $0$ & $0$ & $0$ & $0$\\
	$1$ & $0$ & $0$ & $1$ & $1$ & $0$ & $1$\\
	$1$ & $0$ & $1$ & $1$ & $0$ & $0$ & $1$\\
	$1$ & $1$ & $0$ & $1$ & $1$ & $1$ & $1$\\
	$1$ & $1$ & $1$ & $1$ & $0$ & $1$ & $1$\\
\end{tabular}\\
$K_1$ - простой импликант\\
$K_2$ - не импликант\\
$K_3$ - импликант\\


\subsection{Сокращенные ДНФ}
\subsection{Тупиковые ДНФ}
\subsection{Кратчайшие и минимальные ДНФ}

\section{Классы булевых функций и полнота}
\subsection{Классы БФ}
\subsection{Теорема о функциональной полноте} 
 
\chapter{Теория графов}
\section{Основные понятия}
\textbf{Граф} — абстрактный математический объект, представляющий собой множество вершин графа и набор рёбер, то есть соединений между парами вершин. Например, за множество вершин можно взять множество аэропортов, обслуживаемых некоторой авиакомпанией, а за множество рёбер взять регулярные рейсы этой авиакомпании между городами.

\begin{figure}
	\begin{center}
		\begin{tikzpicture}
		\GraphInit[vstyle=Welsh]
		\SetGraphUnit{2}
		\Vertices{circle}{A,B,C,D}
		\Edges(A,B,C,D,A,C)
		\SetVertexNoLabel
		\end{tikzpicture}
	\end{center}
	\caption{Пример графа из 4 вершин}
\end{figure}

Формальное определение:

\textbf{Графом} называется пара множеств $G = (V, E)$, где $V$~--- множество вершин графа, $E \subseteq V^2$~--- множество рёбер графа.

Если $e = \{ u, v \}$, $e \in E$, то говорят, что:
\begin{itemize}
	\item ребро~$e$ соединяет вершины~$u$ и~$v$;
	\item $u$ и~$v$~--- концы ребра~$e$;
	\item ребро~$e$ инцидентно вершинам $u$ и~$v$;
	\item вершины $u$ и~$v$ \textbf{инцидентны} ребру~$e$.
\end{itemize}

На рисунках вершины графа изображают точками, а~рёбра $e = \{ u, v \}$~--- кривыми, соединяющими точки, которые изображают вершины $u$ и~$v$.

Вершины называются \textbf{соседними}, если их соединяет ребро, иначе~--- \textbf{несоседними}.

Ребро вида $e = \{ u, u \}$ называется \textbf{петлёй}.

Граф, в~котором любые две вершины соединены ребром, называется \textbf{полным} и~обозначается $K_n$, где $n$~--- число вершин в~нём.

Число рёбер в~графе~$G$, инцидентных вершине~$u$, называется \textbf{степенью} вершины и~обозначается $\deg_G u$.


\begin{lemma}[о рукопожатиях]
	\[ \sum_{u \in V} \deg_G u = 2|E| \]
	где $G = (V, E)$~--- граф.
\end{lemma}
\begin{proofmathind}
	\indbase $|E| = 0$: в~таком графе $\displaystyle \sum_{u \in V} \deg u = 0$.
	\indstep Пусть лемма верна для~$|E| = n$.
	Докажем её для~$|E| = n + 1$.
	Для~этого достаточно заметить, что каждое новое ребро увеличивает степени двух вершин на~1.
	\indend
\end{proofmathind}


\textbf{Маршрутом} в~графе~$G = (V, E)$ называется последовательность вершин и~рёбер вида $v_1 e_1 v_2 \ldots e_k v_{k+1}$, где $e_i = \{ v_i, v_{i+1} \}$.

Маршрут, в~котором все рёбра различны, называется \textbf{цепью}.

Цепь, в~которой все вершины, за~исключением, может быть, первой и~последней, различны, называется \textbf{простой}.

Маршрут, в~котором первая и~последняя вершины совпадают, называется \textbf{замкнутым}.

Замкнутая цепь называется \textbf{циклом}.

Маршрут, соединяющий вершины $u$ и~$v$, называется \textbf{$(u, v)$\nobreakdash-\hspace{0pt}маршрутом}.

\begin{lemma}
	$(u, v)$\nobreakdash-\hspace{0pt}маршрут содержит $(u, v)$\nobreakdash-\hspace{0pt}простую цепь.
\end{lemma}
\begin{proof}
	Пусть $u = v_1 e_1 v_2 \ldots e_k v_{k+1} = v$~--- не~простая цепь, тогда $\exists i < j \colon v_i = v_j$.
	Уберём из~маршрута подпоследовательность $e_i v_{i+1} \ldots e_{j-1} v_j$, получим маршрут, в~котором совпадающих вершин на~одну меньше.
	Повторяя, получим простую цепь, являющуюся частью данного маршрута.
\end{proof}

\begin{lemma}
	Любой цикл содержит простой цикл.
\end{lemma}%
Доказательство аналогично предыдущему.

\begin{lemma}
	Если в~графе есть две различные простые цепи, соединяющие одни и~те~же вершины, то в~этом графе есть простой цикл.
\end{lemma}
\begin{proof}
	Пусть $u = v_1 e_1 v_2 \ldots e_n v_{n+1} = v$, $u = v_1' e_1' v_2' \ldots e_m' v_{m+1}' = v$~--- простые цепи.
	Найдём наименьшее~$i \colon e_i \neq e_i'$, тогда $v_i e_i v_{i+1} \ldots e_n v_{n+1} = v_{m+1}' e_m' \ldots e_i' v_i' = v_i$~--- цикл, значит, можно получить простой цикл.
\end{proof}

Графы $G_1 = (V_1, E_1)$ и~$G_2 = (V_2, E_2)$ называются \textbf{изоморфными}, если существует биекция $\varphi \colon V_1 \to V_2$ такая, что
$\forall u, v \in V_1 \ \allowbreak ((u, v) \in E_1 \opbr\Leftrightarrow (\varphi(u), \varphi(v)) \in E_2)$, иначе~--- \textbf{неизоморфными}. Иными словами два графа называются \textbf{изоморфными}, если они одинаковые с точностью до переименования вершин.

$\varphi$ называется \textbf{изоморфизмом}.

На рисунке \ref{isograph} показаны два изоморфных графа $G$ и $G'$. Их изоморфизм $\varphi$ задаётся так:

\begin{itemize}
	\item $A \quad \Leftrightarrow \quad 3$
	\item $B \quad \Leftrightarrow \quad 5$
	\item $C \quad \Leftrightarrow \quad 1$
	\item $D \quad \Leftrightarrow \quad 2$
	\item $E \quad \Leftrightarrow \quad 6$
	\item $F \quad \Leftrightarrow \quad 4$
\end{itemize}

\begin{figure}[h!]
\begin{center}
	\begin{tikzpicture}
	\GraphInit[vstyle=Welsh]
	\SetGraphUnit{2}
	\Vertices{circle}{A,B,C,D,E,F}
	\Edges(A,B,C,D,E,F,B)
	\Edges(A,D)
	\Edges(A,E)
	\Edges(A,C)
	\SetVertexNoLabel
	\end{tikzpicture}
	\begin{tikzpicture}
	\GraphInit[vstyle=Welsh]
	\SetGraphUnit{2}
	\Vertices{circle}{1,2,3,4,5,6}
	\Edges(1,2,3,5,4,6,1,5,3,6,2)
	\SetVertexNoLabel
	\end{tikzpicture}
\end{center}
\caption{Граф $G$ и изоморфный ему граф $G'$}
\label{isograph}
\end{figure}



 % основные определения
\section{Эйлеровы и Гамильтоновы пути и циклы}


\textbf{Эйлеров путь} — это путь, проходящий по всем рёбрам графа и притом только по одному разу. \\
\textbf{Эйлеров цикл} — эйлеров путь, являющийся циклом. То есть замкнутый путь, проходящий через каждое ребро графа ровно по одному разу.\\
\textbf{Эйлеров граф} — граф, содержащий эйлеров цикл.\\
\textbf{Полуэйлеров граф} — граф, содержащий эйлеров путь.\\
\begin{center}
	\textbf{Существование эйлерова цикла и эйлерова пути}
\end{center}
\begin{itemize}
	\item В неориентированном графе\\
	Согласно теореме, доказанной Эйлером, эйлеров цикл существует тогда и только тогда, когда граф связный и в нём отсутствуют вершины нечётной степени.
	
	Эйлеров путь в графе существует тогда и только тогда, когда граф связный и содержит не более двух вершин нечётной степени.[1][2] Ввиду леммы о рукопожатиях, число вершин с нечётной степенью должно быть четным. А значит эйлеров путь существует только тогда, когда это число равно нулю или двум. Причём когда оно равно нулю, эйлеров путь вырождается в эйлеров цикл.
\end{itemize}


\textbf{Гамильтоновым циклом} является такой цикл, который проходит через каждую вершину данного графа ровно по одному разу.\\
\textbf{Гамильтонов граф} - граф, содержащий гамильтонов цикл.\\
\begin{center}
	\textbf{Условия существования гамильтонова цикла в графе}
\end{center}
\begin{itemize}
	\item \textbf{Условие Дирака}\\
	Пусть $p$ — число вершин в данном графе и $p>3$. Если степень каждой вершины не меньше, чем $\frac{p}{2}$, то данный граф — гамильтонов. 
	
	\item \textbf{Условие Оре}\\	
	Пусть $p$ — количество вершин в данном графе и $p>2$. Если для любой пары несмежных вершин $(x, y)$ выполнено неравенство $\deg x + \deg y\geqslant p$, то данный граф — гамильтонов (другими словами: сумма степеней любых двух несмежных вершин не меньше общего числа вершин в графе).
\end{itemize}
 % связанность графов
\input{./mh/discrete_mathematics/graph/trees} % деревья
\section{Планарные графы}
\textbf{Плоский граф} - граф, который "нарисован" на плоскости так, чтобы ребра не пересекались.\\
\textbf{Планарный граф} - граф, который изоморфный плоскому.\\
\textbf{Грань плоского графа} - часть плоскости, границей которого являются его рёбра, и не содержащая внутри себя простых циклов.\\
\begin{center}\textbf{Формула Эйлера для плоских графов.}\end{center}
Если граф плоский, то выполняется такой равенство
$$
\mathbf{n - m + f = 2}
$$
где $n$ - число вершин, $m$ - число рёбер, $f$ - число граней.\\
Для любых планарных выполняется
$$
\mathbf{m \leq 3n - 6}
$$
где $n$ - число вершин, $m$ - число рёбер.\\
\textbf{Разбиением} графа $G$ называется граф, получающийся добавлением новой вершины на какое-нибудь ребро графа $G$.\\
Два графа называются гомеоморфными если получаются разбиением из одного и того же графа. (Стягиваем вершины степени 2 в ребро (удаляем их))\\
\begin{center}Критерий планарности. \textbf{Теорема Понтрягина-Куратовского}\end{center}
\textbf{Граф планарный тогда и только тогда, когда он не содержит подграфов, гомеоморфных $K_5$, $K_{3,3}$.} % планарные графы 

\part{Теория множеств}             
\chapter{Основные понятия}
\section{Определение}
\index{множество}\textbf{Множество} - ключевое понятие теории множест. Оно аксиоматично, то есть неопределяемо. Обозначаются множества обычно заглавными буквами латинского алфивита.\\
 $\in$ - символ принадлежности множеству.\\
\textbf{Пустое множество} — множество, не содержащее ни одного элемента.\\
\textbf{Универсальное множество (универсум)} — множество, содержащее все мыслимые объекты. В связи с парадоксом Рассела данное понятие трактуется в настоящее время как «множество, включающее все множества, участвующие в рассматриваемой задаче».
\section{Аксиоматика}

\section{Операции на множествами}
\begin{itemize}
	\item Объеденение $A \cup B = C \quad \Leftrightarrow \quad \forall c \in C : c \in A \text{ или } c \in B$
	\item Пересечение $A \cap B = C \quad \Leftrightarrow \quad \forall c \in C : c \in A \text{ и } c \in B$
	\item Разница $A \backslash B = C \quad \Leftrightarrow \quad \forall c \in C : c \in A \text{ и } c \notin B$
	\item Симметрическая разница $A \vartriangle B = C \quad \Leftrightarrow \quad \forall c \in C : c \in A \cup \text{ и } c \notin A \cap B $
	\item Дополнение к множеству $A$ (в универсальном множестве  $M$) $\overline{A} \quad \Leftrightarrow \quad \forall x \in \overline{A}, x \in M : x \notin A$
\end{itemize}

          
\chapter{Функции над множествами}
\section{Определение}
\index{функция}\textbf{Функцией $\mathbf{f : A \rightarrow B}$} называется правило, ставящие в соответствие каждому элементу множества $A$ единственный элемент множества $B$ ($f(a) \in B$, $a \in A$).\\
Множество $A$ - \textbf{область определения $f$}.\\
Множество $B$ - \textbf{область заначения $f$}.\\
\section{Биекции, Иньекции, Сюрьекции}
\begin{itemize}
	\item $f : A \rightarrow B$ называют \textbf{иньективной}, если $\forall x, y \in A : x \neq y \Rightarrow f(x) \neq f(y)$
	\item $f : A \rightarrow B$ называют \textbf{сюрьективной}, если $\forall b \in B \exists a \in A : f(a) = b$
	\item $f : A \rightarrow B$ называют \textbf{биньюктивной}, если она является и иньективной и сюрьективной одновременно. 
\end{itemize}


\part{Комбинаторика}       
\hypertarget{pigeonhole_principle}{\section{Принцип Дирихле}}
В комбинаторике \textbf{принцип Дирихле} — утверждение, сформулированное немецким математиком Дирихле в 1834 году, устанавливающее связь между объектами («кроликами») и контейнерами («клетками») при выполнении определённых условий. 

Наиболее тривиальная формулировка принципа Дирихле звучит так:
\begin{center}
\textit{Если кролики рассажены в клетки, причём число кроликов больше числа клеток, то хотя бы в одной из клеток находится более одного кролика.}
\end{center} % Принцип Дирихле

\end{document}\right) 